\subsubsection{operator<<}
\begin{mdframed}
Stream operator, write the vector into a stream
\begin{lstlisting}[language=C++]
template<typename T, uint32_t N> std::ostream& operator<<(std::ostream& stream, const BaseVector<T,N>& v) 
\end{lstlisting}
\textbf{Template Parameters} \\ 
\textit{T} The type of vector \\ 
\textit{N} The size of the vector \\ 
\textbf{Parameters} \\ 
\textit{stream} The stream \\ 
\textit{v} The vector \\ 
\textbf{Return} \\ 
The stream with the modifications\\ 
\end{mdframed}

\subsubsection{cast}
\begin{mdframed}
Cast a vector of type 1 to another vector of type 2
\begin{lstlisting}[language=C++]
template<typename T1, typename T2, uint32_t N> BaseVector<T2,N> cast(const BaseVector<T1,N>& v) 
\end{lstlisting}
\textbf{Template Parameters} \\ 
\textit{T1} Current type \\ 
\textit{T2} New type \\ 
\textit{N} Size of vector \\ 
\textbf{Parameters} \\ 
\textit{v} The current vector \\ 
\textbf{Return} \\ 
The vector of the new type\\ 
\end{mdframed}

\subsubsection{min}
\begin{mdframed}
Get the minimum value of a vector
\begin{lstlisting}[language=C++]
template<typename T, uint32_t N> T min(const BaseVector<T,N>& v) 
\end{lstlisting}
\textbf{Template Parameters} \\ 
\textit{T} Type of vector \\ 
\textit{N} Size of vector \\ 
\textbf{Parameters} \\ 
\textit{v} The vector \\ 
\textbf{Return} \\ 
The smallest component\\ 
\end{mdframed}

\begin{mdframed}
Get the minimum between components and a scalar
\begin{lstlisting}[language=C++]
template<typename T, uint32_t N> BaseVector<T,N> min(BaseVector<T,N> v, T a) 
\end{lstlisting}
\textbf{Template Parameters} \\ 
\textit{T} Type of vector \\ 
\textit{N} Size of vector \\ 
\textbf{Parameters} \\ 
\textit{v} The vector \\ 
\textit{a} The scalar \\ 
\textbf{Return} \\ 
A vector with minimum between the component and the scalar\\ 
\end{mdframed}

\begin{mdframed}
Get the minimum for each components of two vectors
\begin{lstlisting}[language=C++]
template<typename T, uint32_t N> BaseVector<T,N> min(BaseVector<T,N> v1, const BaseVector<T,N>& v2) 
\end{lstlisting}
\textbf{Template Parameters} \\ 
\textit{T} Type of vectors \\ 
\textit{N} Size of vectors \\ 
\textbf{Parameters} \\ 
\textit{v1} First vector \\ 
\textit{v2} Second vector \\ 
\textbf{Return} \\ 
A vector containing the smallest values between the components of the two vectors\\ 
\end{mdframed}

\subsubsection{max}
\begin{mdframed}
Get the maximum value of a vector
\begin{lstlisting}[language=C++]
template<typename T, uint32_t N> T max(const BaseVector<T,N>& v) 
\end{lstlisting}
\textbf{Template Parameters} \\ 
\textit{T} Type of vector \\ 
\textit{N} Size of vector \\ 
\textbf{Parameters} \\ 
\textit{v} The vector \\ 
\textbf{Return} \\ 
The biggest component\\ 
\end{mdframed}

\begin{mdframed}
Get the maximum between components and a scalar
\begin{lstlisting}[language=C++]
template<typename T, uint32_t N> BaseVector<T,N> max(BaseVector<T,N> v, T a) 
\end{lstlisting}
\textbf{Template Parameters} \\ 
\textit{T} Type of vector \\ 
\textit{N} Size of vector \\ 
\textbf{Parameters} \\ 
\textit{v} The vector \\ 
\textit{a} The scalar \\ 
\textbf{Return} \\ 
A vector with maximum between the component and the scalar\\ 
\end{mdframed}

\begin{mdframed}
Get the maximum for each components of two vectors
\begin{lstlisting}[language=C++]
template<typename T, uint32_t N> BaseVector<T,N> max(BaseVector<T,N> v1, const BaseVector<T,N>& v2) 
\end{lstlisting}
\textbf{Template Parameters} \\ 
\textit{T} Type of vectors \\ 
\textit{N} Size of vectors \\ 
\textbf{Parameters} \\ 
\textit{v1} First vector \\ 
\textit{v2} Second vector \\ 
\textbf{Return} \\ 
A vector containing the biggest values between the components of the two vectors\\ 
\end{mdframed}

\subsubsection{abs}
\begin{mdframed}
Return a vector with absolute components
\begin{lstlisting}[language=C++]
template<typename T, uint32_t N> BaseVector<T,N> abs(const BaseVector<T,N>& v) 
\end{lstlisting}
\textbf{Template Parameters} \\ 
\textit{T} Type of vector \\ 
\textit{N} Size of vector \\ 
\textbf{Parameters} \\ 
\textit{v} The vector \\ 
\textbf{Return} \\ 
The vector with absolute component\\ 
\end{mdframed}

\subsubsection{sign}
\begin{mdframed}
Return a vector of +1 or -1 whether the component is > 0 or < 0
\begin{lstlisting}[language=C++]
template<typename T, uint32_t N> BaseVector<T,N> sign(BaseVector<T,N> v) 
\end{lstlisting}
\textbf{Template Parameters} \\ 
\textit{T} Type of vector \\ 
\textit{N} Size of vector \\ 
\textbf{Parameters} \\ 
\textit{v} The vector \\ 
\textbf{Return} \\ 
A vector of the sign of the components\\ 
\end{mdframed}

