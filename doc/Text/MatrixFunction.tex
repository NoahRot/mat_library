\subsubsection{operator<<}
\begin{mdframed}
Stream operator, write the matrix into a stream
\begin{lstlisting}[language=C++]
template<typename T, uint32_t N, uint32_t M> std::ostream& operator<<(std::ostream& stream, const Matrix<T,N,M>& m) 
\end{lstlisting}
\textbf{Template Parameters} \\ 
\textit{T} The type of matrix \\ 
\textit{N} Row \\ 
\textit{M} Column \\ 
\textbf{Parameters} \\ 
\textit{stream} The stream \\ 
\textit{m} The matrix \\ 
\textbf{Return} \\ 
The stream with the modifications\\ 
\end{mdframed}

\subsubsection{operator+}
\begin{mdframed}
Add a scalar to each element of the matrix
\begin{lstlisting}[language=C++]
template<typename T, uint32_t N, uint32_t M> Matrix<T,N,M> operator+(Matrix<T,N,M> m1, T a) 
\end{lstlisting}
\textbf{Template Parameters} \\ 
\textit{T} The type of matrix \\ 
\textit{N} Row \\ 
\textit{M} Column \\ 
\textbf{Parameters} \\ 
\textit{m1} The matrix \\ 
\textit{a} The scalar \\ 
\textbf{Return} \\ 
The new matrix\\ 
\end{mdframed}

\begin{mdframed}
Add a scalar to each element of the matrix
\begin{lstlisting}[language=C++]
template<typename T, uint32_t N, uint32_t M> Matrix<T,N,M> operator+(T a, Matrix<T,N,M> m1) 
\end{lstlisting}
\textbf{Template Parameters} \\ 
\textit{T} The type of matrix \\ 
\textit{N} Row \\ 
\textit{M} Column \\ 
\textbf{Parameters} \\ 
\textit{a} The scalar \\ 
\textit{m1} The matrix \\ 
\textbf{Return} \\ 
The new matrix\\ 
\end{mdframed}

\begin{mdframed}
Addition of two matrices
\begin{lstlisting}[language=C++]
template<typename T, uint32_t N, uint32_t M> Matrix<T,N,M> operator+(Matrix<T,N,M> m1, const Matrix<T,N,M>& m2) 
\end{lstlisting}
\textbf{Template Parameters} \\ 
\textit{T} The type of matrix \\ 
\textit{N} Row \\ 
\textit{M} Column \\ 
\textbf{Parameters} \\ 
\textit{m1} The first matrix \\ 
\textit{m2} The second matrix \\ 
\textbf{Return} \\ 
The addition of the matrices\\ 
\end{mdframed}

\subsubsection{operator-}
\begin{mdframed}
Substract a scalar to each element of the matrix
\begin{lstlisting}[language=C++]
template<typename T, uint32_t N, uint32_t M> Matrix<T,N,M> operator-(Matrix<T,N,M> m1, T a) 
\end{lstlisting}
\textbf{Template Parameters} \\ 
\textit{T} The type of matrix \\ 
\textit{N} Row \\ 
\textit{M} Column \\ 
\textbf{Parameters} \\ 
\textit{m1} The matrix \\ 
\textit{a} The scalar \\ 
\textbf{Return} \\ 
The new matrix\\ 
\end{mdframed}

\begin{mdframed}
Substract element of the matrix by a scalar
\begin{lstlisting}[language=C++]
template<typename T, uint32_t N, uint32_t M> Matrix<T,N,M> operator-(T a, Matrix<T,N,M> m1) 
\end{lstlisting}
\textbf{Template Parameters} \\ 
\textit{T} The type of matrix \\ 
\textit{N} Row \\ 
\textit{M} Column \\ 
\textbf{Parameters} \\ 
\textit{a} The scalar \\ 
\textit{m1} The matrix \\ 
\textbf{Return} \\ 
The new matrix\\ 
\end{mdframed}

\begin{mdframed}
Substraction of two matrices
\begin{lstlisting}[language=C++]
template<typename T, uint32_t N, uint32_t M> Matrix<T,N,M> operator-(Matrix<T,N,M> m1, const Matrix<T,N,M>& m2) 
\end{lstlisting}
\textbf{Template Parameters} \\ 
\textit{T} The type of matrix \\ 
\textit{N} Row \\ 
\textit{M} Column \\ 
\textbf{Parameters} \\ 
\textit{m1} The first matrix \\ 
\textit{m2} The second matrix \\ 
\textbf{Return} \\ 
The substraction of the matrices\\ 
\end{mdframed}

\subsubsection{operator*}
\begin{mdframed}
Multiply each component of a matrix by a scalar
\begin{lstlisting}[language=C++]
template<typename T, uint32_t N, uint32_t M> Matrix<T,N,M> operator*(Matrix<T,N,M> m1, T a) 
\end{lstlisting}
\textbf{Template Parameters} \\ 
\textit{T} The type of matrix \\ 
\textit{N} Row \\ 
\textit{M} Column \\ 
\textbf{Parameters} \\ 
\textit{m1} The matrix \\ 
\textit{a} The scalar \\ 
\textbf{Return} \\ 
The new matrix\\ 
\end{mdframed}

\begin{mdframed}
Multiply each component of a matrix by a scalar
\begin{lstlisting}[language=C++]
template<typename T, uint32_t N, uint32_t M> Matrix<T,N,M> operator*(T a, Matrix<T,N,M> m1) 
\end{lstlisting}
\textbf{Template Parameters} \\ 
\textit{T} The type of matrix \\ 
\textit{N} Row \\ 
\textit{M} Column \\ 
\textbf{Parameters} \\ 
\textit{a} The scalar \\ 
\textit{m1} The matrix \\ 
\textbf{Return} \\ 
The new matrix\\ 
\end{mdframed}

\begin{mdframed}
Multiplication of each elements of two matrices
\begin{lstlisting}[language=C++]
template<typename T, uint32_t N, uint32_t M> Matrix<T,N,M> operator*(Matrix<T,N,M> m1, const Matrix<T,N,M>& m2) 
\end{lstlisting}
\textbf{Template Parameters} \\ 
\textit{T} The type of matrix \\ 
\textit{N} Row \\ 
\textit{M} Column \\ 
\textbf{Parameters} \\ 
\textit{m1} The first matrix \\ 
\textit{m2} The second matrix \\ 
\textbf{Return} \\ 
The multiplication of each element of the matrices\\ 
\textbf{Warning} \\ 
This is not the matrix product. Use dot function for the matrix product\\ 
\end{mdframed}

\subsubsection{operator/}
\begin{mdframed}
Divide each component of a matrix by a scalar
\begin{lstlisting}[language=C++]
template<typename T, uint32_t N, uint32_t M> Matrix<T,N,M> operator/(Matrix<T,N,M> m1, T a) 
\end{lstlisting}
\textbf{Template Parameters} \\ 
\textit{T} The type of matrix \\ 
\textit{N} Row \\ 
\textit{M} Column \\ 
\textbf{Parameters} \\ 
\textit{m1} The matrix \\ 
\textit{a} The scalar \\ 
\textbf{Return} \\ 
The new matrix\\ 
\end{mdframed}

\begin{mdframed}
Division of each elements of two matrices
\begin{lstlisting}[language=C++]
template<typename T, uint32_t N, uint32_t M> Matrix<T,N,M> operator/(Matrix<T,N,M> m1, const Matrix<T,N,M>& m2) 
\end{lstlisting}
\textbf{Template Parameters} \\ 
\textit{T} The type of matrix \\ 
\textit{N} Row \\ 
\textit{M} Column \\ 
\textbf{Parameters} \\ 
\textit{m1} The first matrix \\ 
\textit{m2} The second matrix \\ 
\textbf{Return} \\ 
The division of each element of the matrices\\ 
\textbf{Warning} \\ 
This is not the matrix product. Use dot function for the matrix product\\ 
\end{mdframed}

\subsubsection{min}
\begin{mdframed}
Get the minimum value inside the matrix
\begin{lstlisting}[language=C++]
template<typename T, uint32_t N, uint32_t M> T min(const Matrix<T,N,M>& m) 
\end{lstlisting}
\textbf{Template Parameters} \\ 
\textit{T} The type of matrix \\ 
\textit{N} Row \\ 
\textit{M} Column \\ 
\textbf{Parameters} \\ 
\textit{m} The matrix \\ 
\textbf{Return} \\ 
The minimum component in the matrix\\ 
\end{mdframed}

\begin{mdframed}
Compare each component with a value and return a matrix containing the minimum between component and value
\begin{lstlisting}[language=C++]
template<typename T, uint32_t N, uint32_t M> Matrix<T,N,M> min(Matrix<T,N,M> m, T a) 
\end{lstlisting}
\textbf{Template Parameters} \\ 
\textit{T} The type of matrix \\ 
\textit{N} Row \\ 
\textit{M} Column \\ 
\textbf{Parameters} \\ 
\textit{m} The matrix \\ 
\textit{a} The value \\ 
\textbf{Return} \\ 
A matrix containing the minimum between the component and the value\\ 
\end{mdframed}

\begin{mdframed}
Compare each component and return a matrix containing the minimum value in each component
\begin{lstlisting}[language=C++]
template<typename T, uint32_t N, uint32_t M> Matrix<T,N,M> min(Matrix<T,N,M> m1, const Matrix<T,N,M>& m2) 
\end{lstlisting}
\textbf{Template Parameters} \\ 
\textit{T} The type of matrix \\ 
\textit{N} Row \\ 
\textit{M} Column \\ 
\textbf{Parameters} \\ 
\textit{m1} The first matrix \\ 
\textit{m2} The second matrix \\ 
\textbf{Return} \\ 
A matrix containing the minimum between the component of each matrix\\ 
\end{mdframed}

\subsubsection{max}
\begin{mdframed}
Get the maximum value inside the matrix
\begin{lstlisting}[language=C++]
template<typename T, uint32_t N, uint32_t M> T max(const Matrix<T,N,M>& m) 
\end{lstlisting}
\textbf{Template Parameters} \\ 
\textit{T} The type of matrix \\ 
\textit{N} Row \\ 
\textit{M} Column \\ 
\textbf{Parameters} \\ 
\textit{m} The matrix \\ 
\textbf{Return} \\ 
The maximum component in the matrix\\ 
\end{mdframed}

\begin{mdframed}
Compare each component with a value and return a matrix containing the maximum between component and value
\begin{lstlisting}[language=C++]
template<typename T, uint32_t N, uint32_t M> Matrix<T,N,M> max(Matrix<T,N,M> m, T a) 
\end{lstlisting}
\textbf{Template Parameters} \\ 
\textit{T} The type of matrix \\ 
\textit{N} Row \\ 
\textit{M} Column \\ 
\textbf{Parameters} \\ 
\textit{m} The matrix \\ 
\textit{a} The value \\ 
\textbf{Return} \\ 
A matrix containing the maximum between the component and the value\\ 
\end{mdframed}

\begin{mdframed}
Compare each component and return a matrix containing the maximum value in each component
\begin{lstlisting}[language=C++]
template<typename T, uint32_t N, uint32_t M> Matrix<T,N,M> max(Matrix<T,N,M> m1, const Matrix<T,N,M>& m2) 
\end{lstlisting}
\textbf{Template Parameters} \\ 
\textit{T} The type of matrix \\ 
\textit{N} Row \\ 
\textit{M} Column \\ 
\textbf{Parameters} \\ 
\textit{m1} The first matrix \\ 
\textit{m2} The second matrix \\ 
\textbf{Return} \\ 
A matrix containing the maximum between the component of each matrix\\ 
\end{mdframed}

\subsubsection{abs}
\begin{mdframed}
Apply absolute value to each component of a matrix
\begin{lstlisting}[language=C++]
template<typename T, uint32_t N, uint32_t M> Matrix<T,N,M> abs(Matrix<T,N,M> m) 
\end{lstlisting}
\textbf{Template Parameters} \\ 
\textit{T} The type of matrix \\ 
\textit{N} Row \\ 
\textit{M} Column \\ 
\textbf{Parameters} \\ 
\textit{m} The matrix \\ 
\textbf{Return} \\ 
A matrix with absolute value apply to each component\\ 
\end{mdframed}

\subsubsection{sign}
\begin{mdframed}
Apply sign function to each component of a matrix
\begin{lstlisting}[language=C++]
template<typename T, uint32_t N, uint32_t M> Matrix<T,N,M> sign(Matrix<T,N,M> m) 
\end{lstlisting}
\textbf{Template Parameters} \\ 
\textit{T} The type of matrix \\ 
\textit{N} Row \\ 
\textit{M} Column \\ 
\textbf{Parameters} \\ 
\textit{m} The matrix \\ 
\textbf{Return} \\ 
A matrix with sign function apply to each component\\ 
\end{mdframed}

\subsubsection{to\_row\_matrix}
\begin{mdframed}
Transform a vector to a row matrix
\begin{lstlisting}[language=C++]
template<typename T, uint32_t N> Matrix<T,1,N> to_row_matrix(const Vector<T,N>& v) 
\end{lstlisting}
\textbf{Template Parameters} \\ 
\textit{T} Type \\ 
\textit{N} Size \\ 
\textbf{Parameters} \\ 
\textit{v} The vector \\ 
\textbf{Return} \\ 
The row matrix\\ 
\end{mdframed}

\subsubsection{to\_column\_matrix}
\begin{mdframed}
Transform a vector to a column matrix
\begin{lstlisting}[language=C++]
template<typename T, uint32_t N> Matrix<T,N,1> to_column_matrix(const Vector<T,N>& v) 
\end{lstlisting}
\textbf{Template Parameters} \\ 
\textit{T} Type \\ 
\textit{N} Size \\ 
\textbf{Parameters} \\ 
\textit{v} The vector \\ 
\textbf{Return} \\ 
The column matrix\\ 
\end{mdframed}

\subsubsection{to\_vector}
\begin{mdframed}
Transform a column matrix to a vector
\begin{lstlisting}[language=C++]
template<typename T, uint32_t N> Vector<T,N> to_vector(const Matrix<T,N,1>& m) 
\end{lstlisting}
\textbf{Template Parameters} \\ 
\textit{T} Type \\ 
\textit{N} Size \\ 
\textbf{Parameters} \\ 
\textit{m} The column matrix \\ 
\textbf{Return} \\ 
The vector\\ 
\end{mdframed}

\begin{mdframed}
Transform a row matrix to a vector
\begin{lstlisting}[language=C++]
template<typename T, uint32_t N> Vector<T,N> to_vector(const Matrix<T,1,N>& m) 
\end{lstlisting}
\textbf{Template Parameters} \\ 
\textit{T} Type \\ 
\textit{N} Size \\ 
\textbf{Parameters} \\ 
\textit{m} The row matrix \\ 
\textbf{Return} \\ 
The vector\\ 
\end{mdframed}

\subsubsection{dot}
\begin{mdframed}
Matrix product
\begin{lstlisting}[language=C++]
template<typename T, uint32_t N, uint32_t M, uint32_t P> Matrix<T,N,P> dot(const Matrix<T,N,M>& m1, const Matrix<T,M,P>& m2) 
\end{lstlisting}
\textbf{Template Parameters} \\ 
\textit{T} Type \\ 
\textit{N} Row of the first matrix \\ 
\textit{M} Column of the first matrix and row of the second matrix \\ 
\textit{P} Column of the second matrix \\ 
\textbf{Parameters} \\ 
\textit{m1} The first matrix \\ 
\textit{m2} The second matrix \\ 
\textbf{Return} \\ 
The result matrix\\ 
\end{mdframed}

\begin{mdframed}
Matrix - vector product
\begin{lstlisting}[language=C++]
template<typename T, uint32_t N, uint32_t M> BaseVector<T,N> dot(const Matrix<T,N,M>& m, const BaseVector<T,M>& v) 
\end{lstlisting}
\textbf{Template Parameters} \\ 
\textit{T} Type \\ 
\textit{N} Row of the matrix \\ 
\textit{M} Column of the matrix and size of the vector \\ 
\textbf{Parameters} \\ 
\textit{m} The matrix \\ 
\textit{v} The vector \\ 
\textbf{Return} \\ 
The result vector\\ 
\end{mdframed}

\begin{mdframed}
Vector - matrix product
\begin{lstlisting}[language=C++]
template<typename T, uint32_t N, uint32_t M> BaseVector<T,M> dot(const BaseVector<T,N>& v, const Matrix<T,N,M>& m) 
\end{lstlisting}
\textbf{Template Parameters} \\ 
\textit{T} Type \\ 
\textit{N} Row of the matrix and size of the vector \\ 
\textit{M} Column of the matrix \\ 
\textbf{Parameters} \\ 
\textit{v} The vector \\ 
\textit{m} The matrix \\ 
\textbf{Return} \\ 
The result vector\\ 
\end{mdframed}

\subsubsection{identity}
\begin{mdframed}
Get the identity matrix
\begin{lstlisting}[language=C++]
template<typename T, uint32_t N> Matrix<T,N,N> identity() 
\end{lstlisting}
\textbf{Template Parameters} \\ 
\textit{T} Type \\ 
\textit{N} Row and column of the matrix \\ 
\textbf{Return} \\ 
The identity matrix\\ 
\end{mdframed}

\subsubsection{transpose}
\begin{mdframed}
Get the transpose of a matrix
\begin{lstlisting}[language=C++]
template<typename T, uint32_t N, uint32_t M> Matrix<T,M,N> transpose(const Matrix<T,N,M>& m) 
\end{lstlisting}
\textbf{Template Parameters} \\ 
\textit{T} Type \\ 
\textit{N} Row of the matrix \\ 
\textit{M} Column of the matrix \\ 
\textbf{Parameters} \\ 
\textit{m} The matrix \\ 
\textbf{Return} \\ 
The transpose matrix\\ 
\end{mdframed}

\subsubsection{self\_transpose}
\begin{mdframed}
Transpose the matrix itself
\begin{lstlisting}[language=C++]
template<typename T, uint32_t N> void self_transpose(Matrix<T,N,N>& m) 
\end{lstlisting}
\textbf{Template Parameters} \\ 
\textit{T} Type \\ 
\textit{N} Row and column of the matrix \\ 
\textbf{Parameters} \\ 
\textit{m} The matrix \\ 
\end{mdframed}

\subsubsection{direct\_sum}
\begin{mdframed}
Direct sum of two matrices
\begin{lstlisting}[language=C++]
template<typename T, uint32_t N, uint32_t M, uint32_t P, uint32_t Q> Matrix<T,N+P,M+Q> direct_sum(const Matrix<T,N,M>& m1, const Matrix<T,P,Q>& m2) 
\end{lstlisting}
\textbf{Template Parameters} \\ 
\textit{T} Type \\ 
\textit{N} Row of the first matrix \\ 
\textit{M} Column of the first matrix \\ 
\textit{P} Row of the second matrix \\ 
\textit{Q} Column of the second matrix \\ 
\textbf{Parameters} \\ 
\textit{m1} The first matrix \\ 
\textit{m2} The second matrix \\ 
\textbf{Return} \\ 
The direct sum of the matrices\\ 
\end{mdframed}

\subsubsection{kronecker\_product}
\begin{mdframed}
Kronecker product of two matrices
\begin{lstlisting}[language=C++]
template<typename T, uint32_t N, uint32_t M, uint32_t P, uint32_t Q> Matrix<T,N*P,M*Q> kronecker_product(const Matrix<T,N,M>& m1, const Matrix<T,P,Q>& m2) 
\end{lstlisting}
\textbf{Template Parameters} \\ 
\textit{T} Type \\ 
\textit{N} Row of the first matrix \\ 
\textit{M} Column of the first matrix \\ 
\textit{P} Row of the second matrix \\ 
\textit{Q} Column of the second matrix \\ 
\textbf{Parameters} \\ 
\textit{m1} The first matrix \\ 
\textit{m2} The second matrix \\ 
\textbf{Return} \\ 
The kronecker product of the matrices\\ 
\end{mdframed}

\subsubsection{determinant}
\begin{mdframed}
Compute the determinant of a matrix
\begin{lstlisting}[language=C++]
template<typename T, uint32_t N> T determinant(Matrix<T,N,N> m) 
\end{lstlisting}
\textbf{Template Parameters} \\ 
\textit{T} Type \\ 
\textit{N} Row and column of the first matrix \\ 
\textbf{Parameters} \\ 
\textit{m} The matrix \\ 
\textbf{Return} \\ 
The determinant of the matrix\\ 
\end{mdframed}

\subsubsection{inverse}
\begin{mdframed}
Compute the inverse of a matrix
\begin{lstlisting}[language=C++]
template<typename T, uint32_t N> Matrix<T,N,N> inverse(Matrix<T,N,N> m) 
\end{lstlisting}
\textbf{Template Parameters} \\ 
\textit{T} Type \\ 
\textit{N} Row and column of the first matrix \\ 
\textbf{Parameters} \\ 
\textit{m} The matrix \\ 
\textbf{Return} \\ 
The inverse of the matrix\\ 
\textbf{Warning} \\ 
Does not check if the determinant is non null\\ 
\end{mdframed}

\subsubsection{expand}
\begin{mdframed}
Expand a matrix (from [N x N] to [N+1 x N+1])
\begin{lstlisting}[language=C++]
template<typename T, uint32_t N> Matrix<T,N+1,N+1> expand(const Matrix<T,N,N>& m) 
\end{lstlisting}
\textbf{Template Parameters} \\ 
\textit{T} Type \\ 
\textit{N} Row and column of the first matrix \\ 
\textbf{Parameters} \\ 
\textit{m} The matrix \\ 
\textbf{Return} \\ 
The expand matrix\\ 
\end{mdframed}

\subsubsection{exp}
\begin{mdframed}
Compute the exponential of a matrix
\begin{lstlisting}[language=C++]
template<typename T, uint32_t N> Matrix<T,N,N> exp(const Matrix<T,N,N>& m, T tolerence = 1e-3, uint32_t max_iteration = 30) 
\end{lstlisting}
\textbf{Template Parameters} \\ 
\textit{T} Type of matrix \\ 
\textit{N} Row and column \\ 
\textbf{Parameters} \\ 
\textit{m} The matrix \\ 
\textit{tolerence} Tolerence of the computation \\ 
\textit{max\_iteration} Maximum number of iterations \\ 
\textbf{Return} \\ 
The exponential of a matrix\\ 
\textbf{Warning} \\ 
Tolerence must be smaller than 1, max\_iteration must be smaller than 33\\ 
\end{mdframed}

