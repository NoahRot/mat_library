\subsection{Vector  class}
Mathematical vector
\begin{lstlisting}[language=C++]
template<typename T, uint32_t N> class Vector 
\end{lstlisting}
\textbf{Template Parameters} \\ 
\textit{T} Type of the vector \\ 
\textit{N} Size of the vector \\ 

\subsubsection{Vector}
\begin{mdframed}
Default constructor, fill the vector with value 0
\begin{lstlisting}[language=C++]
Vector()
\end{lstlisting}
\end{mdframed}

\begin{mdframed}
Constructor, fill the vector with a value
\begin{lstlisting}[language=C++]
Vector(T fill_val)
\end{lstlisting}
\textbf{Parameters} \\ 
\textit{fill\_val} Value that fill the vector \\ 
\end{mdframed}

\begin{mdframed}
Constructor, copy an array
\begin{lstlisting}[language=C++]
Vector(const std::array<T,N>& init_array)
\end{lstlisting}
\textbf{Parameters} \\ 
\textit{init\_array} Initial array \\ 
\end{mdframed}

\begin{mdframed}
Constructor, copy an initializer list
\begin{lstlisting}[language=C++]
Vector(const std::initializer_list<T>& init_list)
\end{lstlisting}
\textbf{Parameters} \\ 
\textit{init\_list} Initializer list \\ 
\end{mdframed}

\subsubsection{get\_type\_object}
\begin{mdframed}
Get the type of the object
\begin{lstlisting}[language=C++]
virtual type_mat get_type_object() const override 
\end{lstlisting}
\textbf{Return} \\ 
The type of the object\\ 
\end{mdframed}

\subsubsection{norm2}
\begin{mdframed}
Compute the square of the norm
\begin{lstlisting}[language=C++]
T norm2() const
\end{lstlisting}
\textbf{Return} \\ 
The square of the norm\\ 
\end{mdframed}

\subsubsection{norm}
\begin{mdframed}
Compute the the norm
\begin{lstlisting}[language=C++]
T norm() const
\end{lstlisting}
\textbf{Return} \\ 
The norm\\ 
\end{mdframed}

\subsubsection{normalize}
\begin{mdframed}
Normalize the vector
\begin{lstlisting}[language=C++]
void normalize()
\end{lstlisting}
\end{mdframed}

\subsubsection{operator BaseVector<T,N>\&}
\begin{mdframed}
Cast to a base vector
\begin{lstlisting}[language=C++]
operator BaseVector<T,N>&()
\end{lstlisting}
\end{mdframed}

\subsubsection{operator+=}
\begin{mdframed}
Sum each vector element with a scalar
\begin{lstlisting}[language=C++]
Vector& operator+=(T a)
\end{lstlisting}
\textbf{Parameters} \\ 
\textit{a} Scalar \\ 
\textbf{Return} \\ 
Reference to the vector\\ 
\end{mdframed}

\begin{mdframed}
Add two vectors
\begin{lstlisting}[language=C++]
Vector& operator+=(const Vector<T,N>& v)
\end{lstlisting}
\textbf{Parameters} \\ 
\textit{v} Vector \\ 
\textbf{Return} \\ 
Reference to the vector\\ 
\end{mdframed}

\subsubsection{operator-=}
\begin{mdframed}
Subtract each vector element with a scalar
\begin{lstlisting}[language=C++]
Vector& operator-=(T a)
\end{lstlisting}
\textbf{Parameters} \\ 
\textit{a} Scalar \\ 
\textbf{Return} \\ 
Reference to the vector\\ 
\end{mdframed}

\begin{mdframed}
Substract two vectors
\begin{lstlisting}[language=C++]
Vector& operator-=(const Vector<T,N>& v)
\end{lstlisting}
\textbf{Parameters} \\ 
\textit{v} Vector \\ 
\textbf{Return} \\ 
Reference to the vector\\ 
\end{mdframed}

\subsubsection{operator*=}
\begin{mdframed}
Multiply each vector element with a scalar
\begin{lstlisting}[language=C++]
Vector& operator*=(T a)
\end{lstlisting}
\textbf{Parameters} \\ 
\textit{a} Scalar \\ 
\textbf{Return} \\ 
Reference to the vector\\ 
\end{mdframed}

\begin{mdframed}
Multiply each element of the two vectors
\begin{lstlisting}[language=C++]
Vector& operator*=(const Vector<T,N>& v)
\end{lstlisting}
\textbf{Parameters} \\ 
\textit{v} Vector \\ 
\textbf{Return} \\ 
Reference to the vector\\ 
\textbf{Warning} \\ 
It is not a dot product, use the function dot\\ 
\end{mdframed}

\subsubsection{operator/=}
\begin{mdframed}
Divide each vector element with a scalar
\begin{lstlisting}[language=C++]
Vector& operator/=(T a)
\end{lstlisting}
\textbf{Parameters} \\ 
\textit{a} Scalar \\ 
\textbf{Return} \\ 
Reference to the vector\\ 
\end{mdframed}

\begin{mdframed}
Divide each element of the two vectors
\begin{lstlisting}[language=C++]
Vector& operator/=(const Vector<T,N>& v)
\end{lstlisting}
\textbf{Parameters} \\ 
\textit{v} Vector \\ 
\textbf{Return} \\ 
Reference to the vector\\ 
\end{mdframed}

