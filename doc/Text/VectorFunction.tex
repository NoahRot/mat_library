\subsubsection{operator<<}
\begin{mdframed}
Stream operator, write the vector into a stream
\begin{lstlisting}[language=C++]
template<typename T, uint32_t N> std::ostream& operator<<(std::ostream& stream, const Vector<T,N>& v) 
\end{lstlisting}
\textbf{Template Parameters} \\ 
\textit{T} The type of vector \\ 
\textit{N} The size of the vector \\ 
\textbf{Parameters} \\ 
\textit{stream} The stream \\ 
\textit{v} The vector \\ 
\textbf{Return} \\ 
The stream with the modifications\\ 
\end{mdframed}

\subsubsection{operator+}
\begin{mdframed}
Add a scalar to each element of the vector
\begin{lstlisting}[language=C++]
template<typename T, uint32_t N> Vector<T,N> operator+(Vector<T,N> v1, T a) 
\end{lstlisting}
\textbf{Template Parameters} \\ 
\textit{T} The type of vector \\ 
\textit{N} The size of the vector \\ 
\textbf{Parameters} \\ 
\textit{v1} The vector \\ 
\textit{a} The scalar \\ 
\textbf{Return} \\ 
The new vector\\ 
\end{mdframed}

\begin{mdframed}
Add a scalar to each element of the vector
\begin{lstlisting}[language=C++]
template<typename T, uint32_t N> Vector<T,N> operator+(T a, Vector<T,N> v1) 
\end{lstlisting}
\textbf{Template Parameters} \\ 
\textit{T} The type of vector \\ 
\textit{N} The size of the vector \\ 
\textbf{Parameters} \\ 
\textit{a} The scalar \\ 
\textit{v1} The vector \\ 
\textbf{Return} \\ 
The new vector\\ 
\end{mdframed}

\begin{mdframed}
Addition of two vectors
\begin{lstlisting}[language=C++]
template<typename T, uint32_t N> Vector<T,N> operator+(Vector<T,N> v1, const Vector<T,N>& v2) 
\end{lstlisting}
\textbf{Template Parameters} \\ 
\textit{T} The type of vector \\ 
\textit{N} The size of the vector \\ 
\textbf{Parameters} \\ 
\textit{v1} First vector \\ 
\textit{v2} Second vector \\ 
\textbf{Return} \\ 
The addition of the two vectors\\ 
\end{mdframed}

\subsubsection{operator-}
\begin{mdframed}
Substract a scalar to each element of the vector
\begin{lstlisting}[language=C++]
template<typename T, uint32_t N> Vector<T,N> operator-(Vector<T,N> v1, T a) 
\end{lstlisting}
\textbf{Template Parameters} \\ 
\textit{T} The type of vector \\ 
\textit{N} The size of the vector \\ 
\textbf{Parameters} \\ 
\textit{v1} The vector \\ 
\textit{a} The scalar \\ 
\textbf{Return} \\ 
The new vector\\ 
\end{mdframed}

\begin{mdframed}
Substract element of the vector by a scalar
\begin{lstlisting}[language=C++]
template<typename T, uint32_t N> Vector<T,N> operator-(T a, Vector<T,N> v1) 
\end{lstlisting}
\textbf{Template Parameters} \\ 
\textit{T} The type of vector \\ 
\textit{N} The size of the vector \\ 
\textbf{Parameters} \\ 
\textit{a} The scalar \\ 
\textit{v1} The vector \\ 
\textbf{Return} \\ 
The new vector\\ 
\end{mdframed}

\begin{mdframed}
Substraction of two vectors
\begin{lstlisting}[language=C++]
template<typename T, uint32_t N> Vector<T,N> operator-(Vector<T,N> v1, const Vector<T,N>& v2) 
\end{lstlisting}
\textbf{Template Parameters} \\ 
\textit{T} The type of vector \\ 
\textit{N} The size of the vector \\ 
\textbf{Parameters} \\ 
\textit{v1} First vector \\ 
\textit{v2} Second vector \\ 
\textbf{Return} \\ 
The substraction of the two vectors\\ 
\end{mdframed}

\begin{mdframed}
Each element is multiply by -1
\begin{lstlisting}[language=C++]
template<typename T, uint32_t N> Vector<T,N> operator-(Vector<T,N> v1) 
\end{lstlisting}
\textbf{Template Parameters} \\ 
\textit{T} The type of vector \\ 
\textit{N} The size of the vector \\ 
\textbf{Parameters} \\ 
\textit{v1} The vector \\ 
\textbf{Return} \\ 
Minus the vector (-v)\\ 
\end{mdframed}

\subsubsection{operator*}
\begin{mdframed}
Multiply each component of a vector by a scalar
\begin{lstlisting}[language=C++]
template<typename T, uint32_t N> Vector<T,N> operator*(Vector<T,N> v1, T a) 
\end{lstlisting}
\textbf{Template Parameters} \\ 
\textit{T} The type of vector \\ 
\textit{N} The size of the vector \\ 
\textbf{Parameters} \\ 
\textit{v1} The vector \\ 
\textit{a} The scalar \\ 
\textbf{Return} \\ 
The new vector\\ 
\end{mdframed}

\begin{mdframed}
Multiply each component of a vector by a scalar
\begin{lstlisting}[language=C++]
template<typename T, uint32_t N> Vector<T,N> operator*(T a, Vector<T,N> v1) 
\end{lstlisting}
\textbf{Template Parameters} \\ 
\textit{T} The type of vector \\ 
\textit{N} The size of the vector \\ 
\textbf{Parameters} \\ 
\textit{a} The scalar \\ 
\textit{v1} The vector \\ 
\textbf{Return} \\ 
The new vector\\ 
\end{mdframed}

\begin{mdframed}
Multiply each element of two vectors
\begin{lstlisting}[language=C++]
template<typename T, uint32_t N> Vector<T,N> operator*(Vector<T,N> v1, const Vector<T,N>& v2) 
\end{lstlisting}
\textbf{Template Parameters} \\ 
\textit{T} The type of vector \\ 
\textit{N} The size of the vector \\ 
\textbf{Parameters} \\ 
\textit{v1} First vector \\ 
\textit{v2} Second vector \\ 
\textbf{Return} \\ 
The multiplication result\\ 
\textbf{Warning} \\ 
It is not a dot product, use the function dot\\ 
\end{mdframed}

\subsubsection{operator/}
\begin{mdframed}
Divide each component of a vector by a scalar
\begin{lstlisting}[language=C++]
template<typename T, uint32_t N> Vector<T,N> operator/(Vector<T,N> v1, T a) 
\end{lstlisting}
\textbf{Template Parameters} \\ 
\textit{T} The type of vector \\ 
\textit{N} The size of the vector \\ 
\textbf{Parameters} \\ 
\textit{v1} The vector \\ 
\textit{a} The scalar \\ 
\textbf{Return} \\ 
The new vector\\ 
\end{mdframed}

\begin{mdframed}
Divide each element of two vectors
\begin{lstlisting}[language=C++]
template<typename T, uint32_t N> Vector<T,N> operator/(Vector<T,N> v1, const Vector<T,N>& v2) 
\end{lstlisting}
\textbf{Template Parameters} \\ 
\textit{T} The type of vector \\ 
\textit{N} The size of the vector \\ 
\textbf{Parameters} \\ 
\textit{v1} First vector \\ 
\textit{v2} Second vector \\ 
\textbf{Return} \\ 
The division result\\ 
\end{mdframed}

\subsubsection{sum}
\begin{mdframed}
Sum all the component of a vector
\begin{lstlisting}[language=C++]
template<typename T, uint32_t N> T sum(const Vector<T,N>& v) 
\end{lstlisting}
\textbf{Template Parameters} \\ 
\textit{T} Type of the vector \\ 
\textit{N} Size of the vector \\ 
\textbf{Parameters} \\ 
\textit{v} The vector \\ 
\textbf{Return} \\ 
The sum of the component of the vector\\ 
\end{mdframed}

\subsubsection{merge}
\begin{mdframed}
Merge two vectors into one vector
\begin{lstlisting}[language=C++]
template<typename T, uint32_t N, uint32_t M> Vector<T, N+M> merge(const Vector<T,N>& v1, const Vector<T,M>& v2) 
\end{lstlisting}
\textbf{Template Parameters} \\ 
\textit{T} Type of the vector \\ 
\textit{N} Size of the first vector \\ 
\textit{M} Size of the second vector \\ 
\textbf{Parameters} \\ 
\textit{v1} First vector \\ 
\textit{v2} Second vector \\ 
\textbf{Return} \\ 
Merged vector\\ 
\end{mdframed}

\subsubsection{append}
\begin{mdframed}
Append a component at the back of a vector
\begin{lstlisting}[language=C++]
template<typename T, uint32_t N> Vector<T,N+1> append(const Vector<T,N>& v, T value) 
\end{lstlisting}
\textbf{Template Parameters} \\ 
\textit{T} Type of the vector \\ 
\textit{N} Size of the vector \\ 
\textbf{Parameters} \\ 
\textit{v} The vector \\ 
\textit{value} The new component \\ 
\textbf{Return} \\ 
Vector with the new element\\ 
\end{mdframed}

\subsubsection{insert}
\begin{mdframed}
Insert a component in a vector
\begin{lstlisting}[language=C++]
template<typename T, uint32_t N> Vector<T,N+1> insert(const Vector<T,N>& v, T value, uint32_t index) 
\end{lstlisting}
\textbf{Template Parameters} \\ 
\textit{T} Type of the vector \\ 
\textit{N} Size of the vector \\ 
\textbf{Parameters} \\ 
\textit{v} The vector \\ 
\textit{value} The new component \\ 
\textit{index} The index where the component will be inserted \\ 
\textbf{Return} \\ 
Vector with the new element\\ 
\end{mdframed}

\subsubsection{remove}
\begin{mdframed}
Remove a component from a vector
\begin{lstlisting}[language=C++]
template<typename T, uint32_t N> Vector<T,N-1> remove(const Vector<T,N>&v, uint32_t index) 
\end{lstlisting}
\textbf{Template Parameters} \\ 
\textit{T} Type of the vector \\ 
\textit{N} Size of the vector \\ 
\textbf{Parameters} \\ 
\textit{v} The vector \\ 
\textit{index} The index of the component that will be removed \\ 
\textbf{Return} \\ 
Vector without the component\\ 
\end{mdframed}

\subsubsection{sub\_vector}
\begin{mdframed}
Create a sub vector from a vector
\begin{lstlisting}[language=C++]
template<typename T, uint32_t N, uint32_t I1, uint32_t I2> Vector<T,I2-I1+1> sub_vector(const Vector<T,N>& v) 
\end{lstlisting}
\textbf{Template Parameters} \\ 
\textit{T} Type of the vector \\ 
\textit{N} Size of the vector \\ 
\textit{I1} First element of the sub vector \\ 
\textit{I2} Last element of the sub vector \\ 
\textbf{Parameters} \\ 
\textit{v} The vector \\ 
\textbf{Return} \\ 
The sub vector\\ 
\end{mdframed}

\subsubsection{dot}
\begin{mdframed}
Compute the dot product between two vectors
\begin{lstlisting}[language=C++]
template<typename T, uint32_t N> T dot(Vector<T,N> v1, const Vector<T,N>& v2) 
\end{lstlisting}
\textbf{Template Parameters} \\ 
\textit{T} Type of the vector \\ 
\textit{N} Size of the vector \\ 
\textbf{Parameters} \\ 
\textit{v1} First vector \\ 
\textit{v2} Second vector \\ 
\textbf{Return} \\ 
Result of the dot product\\ 
\end{mdframed}

\subsubsection{cross}
\begin{mdframed}
Compute the cross product between two vectors of dimension 2
\begin{lstlisting}[language=C++]
template<typename T> T cross(const Vector<T,2>& v1, const Vector<T,2>& v2) 
\end{lstlisting}
\textbf{Template Parameters} \\ 
\textit{T} Type of the vector \\ 
\textbf{Parameters} \\ 
\textit{v1} First vector \\ 
\textit{v2} Second vector \\ 
\textbf{Return} \\ 
Result of the cross product (a scalar)\\ 
\end{mdframed}

\begin{mdframed}
Compute the cross product between two vectors of dimension 3
\begin{lstlisting}[language=C++]
template<typename T> Vector<T,3> cross(const Vector<T,3>& v1, const Vector<T,3>& v2) 
\end{lstlisting}
\textbf{Template Parameters} \\ 
\textit{T} Type of the vector \\ 
\textbf{Parameters} \\ 
\textit{v1} First vector \\ 
\textit{v2} Second vector \\ 
\textbf{Return} \\ 
Result of the cross product (a vector 3)\\ 
\end{mdframed}

\subsubsection{angle}
\begin{mdframed}
Compute the angle between two vectors of dimension 2
\begin{lstlisting}[language=C++]
template<typename T> T angle(const Vector<T,2>& v1, const Vector<T,2>& v2) 
\end{lstlisting}
\textbf{Template Parameters} \\ 
\textit{T} Type of the vector \\ 
\textbf{Parameters} \\ 
\textit{v1} First vector \\ 
\textit{v2} Second vector \\ 
\textbf{Return} \\ 
Angle between the vectors\\ 
\end{mdframed}

\begin{mdframed}
Compute the angle between two vectors of dimension 3
\begin{lstlisting}[language=C++]
template<typename T> T angle(const Vector<T,3>& v1, const Vector<T,3>& v2) 
\end{lstlisting}
\textbf{Template Parameters} \\ 
\textit{T} Type of the vector \\ 
\textbf{Parameters} \\ 
\textit{v1} First vector \\ 
\textit{v2} Second vector \\ 
\textbf{Return} \\ 
Angle between the vectors\\ 
\end{mdframed}

\subsubsection{polar}
\begin{mdframed}
Compute the polar representation of a vector 2
\begin{lstlisting}[language=C++]
template<typename T> void polar(const Vector<T,2>& v, T& radius, T& theta) 
\end{lstlisting}
\textbf{Template Parameters} \\ 
\textit{T} Type of the vector \\ 
\textbf{Parameters} \\ 
\textit{v} The vector \\ 
\textit{radius} A reference to the magnitude \\ 
\textit{theta} A reference to the argument (angle) of the vector \\ 
\end{mdframed}

\subsubsection{cylindrical}
\begin{mdframed}
Compute the cylindrical representation of a vector 3
\begin{lstlisting}[language=C++]
template<typename T> void cylindrical(const Vector<T,3>& v, T& radius, T& theta, T& z) 
\end{lstlisting}
\textbf{Template Parameters} \\ 
\textit{T} Type of the vector \\ 
\textbf{Parameters} \\ 
\textit{v} The vector \\ 
\textit{radius} A reference to the radius of the cylinder \\ 
\textit{theta} A reference to the angle \\ 
\textit{z} A reference to the z coordinate \\ 
\end{mdframed}

\subsubsection{spherical}
\begin{mdframed}
Compute the spherical representation of a vector 3
\begin{lstlisting}[language=C++]
template<typename T> void spherical(const Vector<T,3>& v, T& radius, T& theta, T& phi) 
\end{lstlisting}
\textbf{Template Parameters} \\ 
\textit{T} Type of the vector \\ 
\textbf{Parameters} \\ 
\textit{v} The vector \\ 
\textit{radius} A reference to the radius of the sphere \\ 
\textit{theta} A reference to the theta angle \\ 
\textit{phi} A reference to the phi angle \\ 
\end{mdframed}

\subsubsection{distance}
\begin{mdframed}
Compute the distance between two vectors
\begin{lstlisting}[language=C++]
template<typename T, uint32_t N> T distance(const Vector<T,N>& v1, const Vector<T,N>& v2) 
\end{lstlisting}
\textbf{Template Parameters} \\ 
\textit{T} Type of vector \\ 
\textit{N} Size of vectors \\ 
\textbf{Parameters} \\ 
\textit{v1} First vector \\ 
\textit{v2} Second vector \\ 
\textbf{Return} \\ 
The distance between the vectors\\ 
\end{mdframed}

\subsubsection{reflect}
\begin{mdframed}
Compute the reflected vector
\begin{lstlisting}[language=C++]
template<typename T, uint32_t N> Vector<T,N> reflect(const Vector<T,N>& incident, Vector<T,N> surface_normal) 
\end{lstlisting}
\textbf{Template Parameters} \\ 
\textit{T} Type of vector \\ 
\textit{N} Site of vector \\ 
\textbf{Parameters} \\ 
\textit{incident} The incident ray \\ 
\textit{surface\_normal} The surface normal \\ 
\textbf{Return} \\ 
The refleted vector\\ 
\end{mdframed}

\subsubsection{refract}
\begin{mdframed}
Compute the refracted vector
\begin{lstlisting}[language=C++]
template<typename T, uint32_t N> Vector<T,N> refract(const Vector<T,N>& incident, Vector<T,N> surface_normal, T n1, T n2) 
\end{lstlisting}
\textbf{Template Parameters} \\ 
\textit{T} Type of vector \\ 
\textit{N} Size of vector \\ 
\textbf{Parameters} \\ 
\textit{incident} The incident ray \\ 
\textit{surface\_normal} The surface normal \\ 
\textit{n1} The refraction index of the medium of the incident ray \\ 
\textit{n2} The refraction index of the medium of the surface \\ 
\textbf{Return} \\ 
The refracted vector\\ 
\end{mdframed}

\subsubsection{face\_same\_direction}
\begin{mdframed}
Get if both vector are facing in the same direction (dot(v1, v2) >? 0)
\begin{lstlisting}[language=C++]
template<typename T, uint32_t N> bool face_same_direction(const Vector<T,N>& v1, const Vector<T,N>& v2) 
\end{lstlisting}
\textbf{Template Parameters} \\ 
\textit{T} Type of vector \\ 
\textit{N} Size of vector \\ 
\textbf{Parameters} \\ 
\textit{v1} First vector \\ 
\textit{v2} Second vector \\ 
\textbf{Return} \\ 
True if dot(v1, v2) > 0, false otherwise\\ 
\end{mdframed}

