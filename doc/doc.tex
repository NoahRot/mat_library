% ___      _ _   _      _ _          _   _          
%|_ _|_ _ (_) |_(_)__ _| (_)___ __ _| |_(_)___ _ _  
% | || ' \| |  _| / _` | | (_-</ _` |  _| / _ \ ' \ 
%|___|_||_|_|\__|_\__,_|_|_/__/\__,_|\__|_\___/_||_|                          

\documentclass[a4paper, 12pt,oneside]{article}

% pour l'inclusion de figures en eps,pdf,jpg,....
\usepackage{graphicx}
% Pour que les figures soit cote à cote
\usepackage{subfigure}
% Pour que la figure soit à coté du texte
\usepackage{wrapfig}

%pour que la figure soit à l'endroit souhaiter [H]
\usepackage{float} 

\usepackage{multirow}

\usepackage{ upgreek }

\usepackage{amssymb}

\usepackage[usenames,dvipsnames]{color}
\usepackage{listings}
\usepackage{xcolor}
\usepackage{tcolorbox}
\usepackage{mdframed}


%Marges. Désactiver pour utiliser les valeurs LaTeX par défaut
%\usepackage[top=2.5cm, bottom=2cm, left=2cm, right=2cm, showframe]{geometry}
\usepackage[top=2.5cm, bottom=2cm, left=2cm, right=2cm]{geometry}

% quelques symboles mathematiques en plus
\usepackage{amsmath}
\usepackage{physics}
\usepackage{amsfonts}

% on peut ecrire directement les charactères avec l'accent
\usepackage[T1]{fontenc}

% a utiliser sur Linux/Windows
%\usepackage[latin1]{inputenc}

% a utiliser avec UTF8
\usepackage[utf8]{inputenc}
%Très utiles pour les groupes mixtes mac/PC. Un fichier texte enregistré sous codage UTF-8 est lisible dans les deux environnement.
%Plus de problème de caractères accentués et spéciaux qui ne s'affichent pas

% pour l'inclusion de liens dans le document (pdflatex)
\usepackage[colorlinks,bookmarks=false,linkcolor=black,urlcolor=blue, citecolor=black]{hyperref}

%Pour l'utilisation plus simple des unités et fractions
\usepackage{units}

%Pour utiliser du time new roman... Comenter pour utiliser du ComputerModern
\usepackage{mathptmx}

%Pour du code non interprété
\usepackage{listings}
\usepackage{verbatim}
\usepackage{verbdef}
\usepackage{lipsum}
\usepackage{mathtools}

\usepackage{esvect}

\def \be {\begin{equation}}
\def \ee {\end{equation}}

%Pour changer la taille des titres de section et subsection. Ajoutez manuellement les autres styles si besoin.
\makeatletter
\renewcommand{\section}{\@startsection {section}{1}{\z@}%
             {-3.5ex \@plus -1ex \@minus -.2ex}%
             {2.3ex \@plus.2ex}%
             {\normalfont\normalsize\bfseries}}
\makeatother

\makeatletter
\renewcommand{\subsection}{\@startsection {subsection}{1}{\z@}%
             {-3.5ex \@plus -1ex \@minus -.2ex}%
             {2.3ex \@plus.2ex}%
             {\normalfont\normalsize\bfseries}}
\makeatother

% Prepare code listing package
\definecolor{codegreen}{rgb}{0,0.6,0.2}
\definecolor{codegray}{rgb}{0.5,0.5,0.5}
\definecolor{codepurple}{rgb}{0.8,0,0.8}
\definecolor{backcolour}{rgb}{0.92,0.92,0.95}

\lstdefinestyle{mystyle}{
    backgroundcolor=\color{backcolour},   
    commentstyle=\color{codegreen},
    keywordstyle=\color{magenta},
    numberstyle=\tiny\color{codegray},
    stringstyle=\color{codepurple},
    basicstyle=\ttfamily\footnotesize,
    breakatwhitespace=false,         
    breaklines=true,                 
    captionpos=b,                    
    keepspaces=true,                 
    numbers=left,                    
    numbersep=5pt,                  
    showspaces=false,                
    showstringspaces=false,
    showtabs=false,                  
    tabsize=2
}

\lstset{style=mystyle}

% ___                             _   
%|   \ ___  __ _  _ _ __  ___ _ _| |_ 
%| |) / _ \/ _| || | '  \/ -_) ' \  _|
%|___/\___/\__|\_,_|_|_|_\___|_||_\__|
                                     

\begin{document}
\title{MAT library - Documentation}
\date{\normalsize{\today}}
\author{Rotunno Noah}
\maketitle

\tableofcontents
\newpage

\section{Introduction}
A mathematical library that implement mathematical classes and functions. The header files of the library has to be included in the "include" folder of your application. No dynamicly link library (dll) or static library (lib) are required. The file \textit{Math.hpp} has to be included:

\begin{lstlisting}[language=C++]
#include <Math.hpp>
\end{lstlisting}

Some type definitions are already implemented. To choose if you want to use these type definitions, the macro \textit{MAT\textunderscore TYPEDEF}
 has to be defined to 1  or 0 to respectively use the type definitions or not. This has to be done \textbf{before} including \textit{Math.hpp}. Example :

\begin{lstlisting}[language=C++]
#define MAT_TYPEDEF 1 /* use 0 to not use the type definitions */
#include <Math.hpp>
\end{lstlisting}

\section{Types}
This section contain the types of MatObject, the type of content allowed and the function associated with them,
\subsection{type\_mat  class}
Type of MatObject that exist
\begin{lstlisting}[language=C++]
enum class type_mat 
\end{lstlisting}

\subsection{type\_content  class}
Type of content in the MatObject that are allowed
\begin{lstlisting}[language=C++]
enum class type_content 
\end{lstlisting}

\subsection{type\_traits  structure}
Function to get the type of content (see specialisation below)
\begin{lstlisting}[language=C++]
template<typename T> struct type_traits 
\end{lstlisting}
\textbf{Template Parameters} \\ 
\textit{T} Type of content \\ 

\subsubsection{get\_type}
\begin{mdframed}
Get the type enum value
\begin{lstlisting}[language=C++]
template<typename T> type_content get_type() 
\end{lstlisting}
\textbf{Template Parameters} \\ 
\textit{T} The type \\ 
\textbf{Return} \\ 
The enum value, or unvalid value if not supported type\\ 
\end{mdframed}

\subsubsection{to\_str}
\begin{mdframed}
Transform a type of MatObject in a string
\begin{lstlisting}[language=C++]
inline std::string to_str(type_mat t) 
\end{lstlisting}
\textbf{Parameters} \\ 
\textit{t} Type of MatObject \\ 
\textbf{Return} \\ 
A string refering to the type of the MatObject\\ 
\end{mdframed}

\begin{mdframed}
Transform a type of content in a string
\begin{lstlisting}[language=C++]
inline std::string to_str(type_content t) 
\end{lstlisting}
\textbf{Parameters} \\ 
\textit{t} Type of content \\ 
\textbf{Return} \\ 
A string refering to the type of the content\\ 
\end{mdframed}

\subsubsection{to\_type\_mat}
\begin{mdframed}
Transform a string to the type of MatObject
\begin{lstlisting}[language=C++]
inline type_mat to_type_mat(std::string t) 
\end{lstlisting}
\textbf{Parameters} \\ 
\textit{t} The string \\ 
\textbf{Return} \\ 
The type of MatObject\\ 
\end{mdframed}

\subsubsection{to\_type\_content}
\begin{mdframed}
Transform a string to the type of content
\begin{lstlisting}[language=C++]
inline type_content to_type_content(std::string t) 
\end{lstlisting}
\textbf{Parameters} \\ 
\textit{t} The string \\ 
\textbf{Return} \\ 
The type of content\\ 
\end{mdframed}



\section{Classes}
This section contain the documentation and the prototype of the different classes. There is also the different functions associated to the classes. Each method and function has de brief explanation and the parameters and template parameters are explained. There is also a brief explanation on the return and if the method or function has some warning to consider.

\subsection{MatObject  class}
The base class for all object in the MAT library that need multiple components
\begin{lstlisting}[language=C++]
template<typename T, uint32_t SIZE> class MatObject 
\end{lstlisting}
\textbf{Template Parameters} \\ 
\textit{T} Type of the content \\ 
\textit{SIZE} Number of components in the object \\ 

\subsubsection{MatObject}
\begin{mdframed}
Constructor
\begin{lstlisting}[language=C++]
MatObject()
\end{lstlisting}
\textbf{Warning} \\ 
Throw a runtime error if the content type is unvalid\\ 
\end{mdframed}

\subsubsection{get\_type\_object}
\begin{mdframed}
Get the type of the MatObject
\begin{lstlisting}[language=C++]
virtual type_mat get_type_object() const 
\end{lstlisting}
\textbf{Return} \\ 
Type of the MatObject\\ 
\end{mdframed}

\subsubsection{get\_type\_content}
\begin{mdframed}
Get the type of the content of the Type of the MatObject
\begin{lstlisting}[language=C++]
type_content get_type_content() const 
\end{lstlisting}
\textbf{Return} \\ 
Type of the content in the MatObject\\ 
\end{mdframed}

\subsubsection{get\_array\_size}
\begin{mdframed}
Get the size of the array containing the component of the MatObject
\begin{lstlisting}[language=C++]
uint32_t get_array_size() const 
\end{lstlisting}
\textbf{Return} \\ 
Number of component in the MatObject\\ 
\end{mdframed}

\subsubsection{data}
\begin{mdframed}
Get a constant reference to the array of the MatObject
\begin{lstlisting}[language=C++]
const std::array<T,SIZE>& data() const 
\end{lstlisting}
\textbf{Return} \\ 
A constant reference to the array\\ 
\end{mdframed}

\begin{mdframed}
Get a reference to the array of the MatObject
\begin{lstlisting}[language=C++]
std::array<T,SIZE>& data() 
\end{lstlisting}
\textbf{Return} \\ 
A reference to the array\\ 
\end{mdframed}

\subsubsection{copy}
\begin{mdframed}
Copy data into the MatObject
\begin{lstlisting}[language=C++]
void copy(T* start)
\end{lstlisting}
\textbf{Parameters} \\ 
\textit{start} A pointer to the first element that will be copy \\ 
\end{mdframed}

\subsubsection{operator==}
\begin{mdframed}
Is equal operator (true if all components are equal)
\begin{lstlisting}[language=C++]
bool operator==(const MatObject<T,SIZE>& m) const
\end{lstlisting}
\textbf{Parameters} \\ 
\textit{v} The MatObject that will be compared \\ 
\textbf{Return} \\ 
If the MatObjects are equal\\ 
\end{mdframed}

\subsubsection{operator!=}
\begin{mdframed}
Is different operator (true if at least one component is different)
\begin{lstlisting}[language=C++]
bool operator!=(const MatObject<T,SIZE>& v) const
\end{lstlisting}
\textbf{Parameters} \\ 
\textit{v} The MatObject that will be compared \\ 
\textbf{Return} \\ 
If the MatObjects are different\\ 
\end{mdframed}

\subsubsection{begin}
\begin{mdframed}
Get the begin iterator
\begin{lstlisting}[language=C++]
iterator begin()
\end{lstlisting}
\textbf{Return} \\ 
The begin iterator\\ 
\end{mdframed}

\begin{mdframed}
Get the constant begin iterator
\begin{lstlisting}[language=C++]
const_iterator begin() const
\end{lstlisting}
\textbf{Return} \\ 
The constant begin iterator\\ 
\end{mdframed}

\subsubsection{end}
\begin{mdframed}
Get the end iterator
\begin{lstlisting}[language=C++]
iterator end()
\end{lstlisting}
\textbf{Return} \\ 
The end iterator\\ 
\end{mdframed}

\begin{mdframed}
Get the constant end iterator
\begin{lstlisting}[language=C++]
const_iterator end() const
\end{lstlisting}
\textbf{Return} \\ 
The constant end iterator\\ 
\end{mdframed}

\subsubsection{m\_component}
\begin{mdframed}
Component of the object
\begin{lstlisting}[language=C++]
std::array<T, SIZE> m_component
\end{lstlisting}
\end{mdframed}



\subsection{BaseVector  class}
Class that represent a vector without any operations
\begin{lstlisting}[language=C++]
template<typename T, uint32_t N> class BaseVector 
\end{lstlisting}
\textbf{Template Parameters} \\ 
\textit{T} Type of vector \\ 
\textit{N} Dimension of the vector \\ 

\subsubsection{BaseVector}
\begin{mdframed}
Default constructor (fill with 0)
\begin{lstlisting}[language=C++]
BaseVector()
\end{lstlisting}
\end{mdframed}

\begin{mdframed}
Constructor with an array
\begin{lstlisting}[language=C++]
BaseVector(const std::array<T,N>& init_array)
\end{lstlisting}
\textbf{Parameters} \\ 
\textit{init\_array} Array of the components \\ 
\end{mdframed}

\begin{mdframed}
Constructor with an initializer list
\begin{lstlisting}[language=C++]
BaseVector(const std::initializer_list<T>& init_list)
\end{lstlisting}
\textbf{Parameters} \\ 
\textit{init\_list} Initializer list containing the components \\ 
\end{mdframed}

\subsubsection{get\_type\_object}
\begin{mdframed}
Get the type of the object
\begin{lstlisting}[language=C++]
virtual type_mat get_type_object() const override 
\end{lstlisting}
\textbf{Return} \\ 
The type of the object\\ 
\end{mdframed}

\subsubsection{size}
\begin{mdframed}
Get the size (dimension) of the vector
\begin{lstlisting}[language=C++]
uint32_t size() const
\end{lstlisting}
\textbf{Return} \\ 
The size (dimension) of the vector\\ 
\end{mdframed}

\subsubsection{operator[]}
\begin{mdframed}
Constant accessor operator
\begin{lstlisting}[language=C++]
const T& operator[](uint32_t index) const
\end{lstlisting}
\textbf{Parameters} \\ 
\textit{index} Index of the element \\ 
\textbf{Return} \\ 
A constant reference to the element\\ 
\end{mdframed}

\begin{mdframed}
Accessor operator
\begin{lstlisting}[language=C++]
T& operator[](uint32_t index)
\end{lstlisting}
\textbf{Parameters} \\ 
\textit{index} Index of the element \\ 
\textbf{Return} \\ 
A reference to the element\\ 
\end{mdframed}

\subsubsection{operator Vector<T,N>\&}
\begin{mdframed}
Cast to a vector
\begin{lstlisting}[language=C++]
operator Vector<T,N>&()
\end{lstlisting}
\end{mdframed}

\subsubsection{operator Complex<T>\&}
\begin{mdframed}
Cast to a complex
\begin{lstlisting}[language=C++]
operator Complex<T>&()
\end{lstlisting}
\textbf{Warning} \\ 
Only if N = 2\\ 
\end{mdframed}

\subsubsection{operator Quaternion<T>\&}
\begin{mdframed}
Cast to a quaternion
\begin{lstlisting}[language=C++]
operator Quaternion<T>&()
\end{lstlisting}
\textbf{Warning} \\ 
Only if N = 4\\ 
\end{mdframed}

 

\subsection{BaseVector functions}
\subsubsection{operator<<}
\begin{mdframed}
Stream operator, write the vector into a stream
\begin{lstlisting}[language=C++]
template<typename T, uint32_t N> std::ostream& operator<<(std::ostream& stream, const BaseVector<T,N>& v) 
\end{lstlisting}
\textbf{Template Parameters} \\ 
\textit{T} The type of vector \\ 
\textit{N} The size of the vector \\ 
\textbf{Parameters} \\ 
\textit{stream} The stream \\ 
\textit{v} The vector \\ 
\textbf{Return} \\ 
The stream with the modifications\\ 
\end{mdframed}

\subsubsection{cast}
\begin{mdframed}
Cast a vector of type 1 to another vector of type 2
\begin{lstlisting}[language=C++]
template<typename T1, typename T2, uint32_t N> BaseVector<T2,N> cast(const BaseVector<T1,N>& v) 
\end{lstlisting}
\textbf{Template Parameters} \\ 
\textit{T1} Current type \\ 
\textit{T2} New type \\ 
\textit{N} Size of vector \\ 
\textbf{Parameters} \\ 
\textit{v} The current vector \\ 
\textbf{Return} \\ 
The vector of the new type\\ 
\end{mdframed}

\subsubsection{min}
\begin{mdframed}
Get the minimum value of a vector
\begin{lstlisting}[language=C++]
template<typename T, uint32_t N> T min(const BaseVector<T,N>& v) 
\end{lstlisting}
\textbf{Template Parameters} \\ 
\textit{T} Type of vector \\ 
\textit{N} Size of vector \\ 
\textbf{Parameters} \\ 
\textit{v} The vector \\ 
\textbf{Return} \\ 
The smallest component\\ 
\end{mdframed}

\begin{mdframed}
Get the minimum between components and a scalar
\begin{lstlisting}[language=C++]
template<typename T, uint32_t N> BaseVector<T,N> min(BaseVector<T,N> v, T a) 
\end{lstlisting}
\textbf{Template Parameters} \\ 
\textit{T} Type of vector \\ 
\textit{N} Size of vector \\ 
\textbf{Parameters} \\ 
\textit{v} The vector \\ 
\textit{a} The scalar \\ 
\textbf{Return} \\ 
A vector with minimum between the component and the scalar\\ 
\end{mdframed}

\begin{mdframed}
Get the minimum for each components of two vectors
\begin{lstlisting}[language=C++]
template<typename T, uint32_t N> BaseVector<T,N> min(BaseVector<T,N> v1, const BaseVector<T,N>& v2) 
\end{lstlisting}
\textbf{Template Parameters} \\ 
\textit{T} Type of vectors \\ 
\textit{N} Size of vectors \\ 
\textbf{Parameters} \\ 
\textit{v1} First vector \\ 
\textit{v2} Second vector \\ 
\textbf{Return} \\ 
A vector containing the smallest values between the components of the two vectors\\ 
\end{mdframed}

\subsubsection{max}
\begin{mdframed}
Get the maximum value of a vector
\begin{lstlisting}[language=C++]
template<typename T, uint32_t N> T max(const BaseVector<T,N>& v) 
\end{lstlisting}
\textbf{Template Parameters} \\ 
\textit{T} Type of vector \\ 
\textit{N} Size of vector \\ 
\textbf{Parameters} \\ 
\textit{v} The vector \\ 
\textbf{Return} \\ 
The biggest component\\ 
\end{mdframed}

\begin{mdframed}
Get the maximum between components and a scalar
\begin{lstlisting}[language=C++]
template<typename T, uint32_t N> BaseVector<T,N> max(BaseVector<T,N> v, T a) 
\end{lstlisting}
\textbf{Template Parameters} \\ 
\textit{T} Type of vector \\ 
\textit{N} Size of vector \\ 
\textbf{Parameters} \\ 
\textit{v} The vector \\ 
\textit{a} The scalar \\ 
\textbf{Return} \\ 
A vector with maximum between the component and the scalar\\ 
\end{mdframed}

\begin{mdframed}
Get the maximum for each components of two vectors
\begin{lstlisting}[language=C++]
template<typename T, uint32_t N> BaseVector<T,N> max(BaseVector<T,N> v1, const BaseVector<T,N>& v2) 
\end{lstlisting}
\textbf{Template Parameters} \\ 
\textit{T} Type of vectors \\ 
\textit{N} Size of vectors \\ 
\textbf{Parameters} \\ 
\textit{v1} First vector \\ 
\textit{v2} Second vector \\ 
\textbf{Return} \\ 
A vector containing the biggest values between the components of the two vectors\\ 
\end{mdframed}

\subsubsection{abs}
\begin{mdframed}
Return a vector with absolute components
\begin{lstlisting}[language=C++]
template<typename T, uint32_t N> BaseVector<T,N> abs(const BaseVector<T,N>& v) 
\end{lstlisting}
\textbf{Template Parameters} \\ 
\textit{T} Type of vector \\ 
\textit{N} Size of vector \\ 
\textbf{Parameters} \\ 
\textit{v} The vector \\ 
\textbf{Return} \\ 
The vector with absolute component\\ 
\end{mdframed}

\subsubsection{sign}
\begin{mdframed}
Return a vector of +1 or -1 whether the component is > 0 or < 0
\begin{lstlisting}[language=C++]
template<typename T, uint32_t N> BaseVector<T,N> sign(BaseVector<T,N> v) 
\end{lstlisting}
\textbf{Template Parameters} \\ 
\textit{T} Type of vector \\ 
\textit{N} Size of vector \\ 
\textbf{Parameters} \\ 
\textit{v} The vector \\ 
\textbf{Return} \\ 
A vector of the sign of the components\\ 
\end{mdframed}



\subsection{Vector  class}
Mathematical vector
\begin{lstlisting}[language=C++]
template<typename T, uint32_t N> class Vector 
\end{lstlisting}
\textbf{Template Parameters} \\ 
\textit{T} Type of the vector \\ 
\textit{N} Size of the vector \\ 

\subsubsection{Vector}
\begin{mdframed}
Default constructor, fill the vector with value 0
\begin{lstlisting}[language=C++]
Vector()
\end{lstlisting}
\end{mdframed}

\begin{mdframed}
Constructor, fill the vector with a value
\begin{lstlisting}[language=C++]
Vector(T fill_val)
\end{lstlisting}
\textbf{Parameters} \\ 
\textit{fill\_val} Value that fill the vector \\ 
\end{mdframed}

\begin{mdframed}
Constructor, copy an array
\begin{lstlisting}[language=C++]
Vector(const std::array<T,N>& init_array)
\end{lstlisting}
\textbf{Parameters} \\ 
\textit{init\_array} Initial array \\ 
\end{mdframed}

\begin{mdframed}
Constructor, copy an initializer list
\begin{lstlisting}[language=C++]
Vector(const std::initializer_list<T>& init_list)
\end{lstlisting}
\textbf{Parameters} \\ 
\textit{init\_list} Initializer list \\ 
\end{mdframed}

\subsubsection{get\_type\_object}
\begin{mdframed}
Get the type of the object
\begin{lstlisting}[language=C++]
virtual type_mat get_type_object() const override 
\end{lstlisting}
\textbf{Return} \\ 
The type of the object\\ 
\end{mdframed}

\subsubsection{norm2}
\begin{mdframed}
Compute the square of the norm
\begin{lstlisting}[language=C++]
T norm2() const
\end{lstlisting}
\textbf{Return} \\ 
The square of the norm\\ 
\end{mdframed}

\subsubsection{norm}
\begin{mdframed}
Compute the the norm
\begin{lstlisting}[language=C++]
T norm() const
\end{lstlisting}
\textbf{Return} \\ 
The norm\\ 
\end{mdframed}

\subsubsection{normalize}
\begin{mdframed}
Normalize the vector
\begin{lstlisting}[language=C++]
void normalize()
\end{lstlisting}
\end{mdframed}

\subsubsection{operator BaseVector<T,N>\&}
\begin{mdframed}
Cast to a base vector
\begin{lstlisting}[language=C++]
operator BaseVector<T,N>&()
\end{lstlisting}
\end{mdframed}

\subsubsection{operator+=}
\begin{mdframed}
Sum each vector element with a scalar
\begin{lstlisting}[language=C++]
Vector& operator+=(T a)
\end{lstlisting}
\textbf{Parameters} \\ 
\textit{a} Scalar \\ 
\textbf{Return} \\ 
Reference to the vector\\ 
\end{mdframed}

\begin{mdframed}
Add two vectors
\begin{lstlisting}[language=C++]
Vector& operator+=(const Vector<T,N>& v)
\end{lstlisting}
\textbf{Parameters} \\ 
\textit{v} Vector \\ 
\textbf{Return} \\ 
Reference to the vector\\ 
\end{mdframed}

\subsubsection{operator-=}
\begin{mdframed}
Subtract each vector element with a scalar
\begin{lstlisting}[language=C++]
Vector& operator-=(T a)
\end{lstlisting}
\textbf{Parameters} \\ 
\textit{a} Scalar \\ 
\textbf{Return} \\ 
Reference to the vector\\ 
\end{mdframed}

\begin{mdframed}
Substract two vectors
\begin{lstlisting}[language=C++]
Vector& operator-=(const Vector<T,N>& v)
\end{lstlisting}
\textbf{Parameters} \\ 
\textit{v} Vector \\ 
\textbf{Return} \\ 
Reference to the vector\\ 
\end{mdframed}

\subsubsection{operator*=}
\begin{mdframed}
Multiply each vector element with a scalar
\begin{lstlisting}[language=C++]
Vector& operator*=(T a)
\end{lstlisting}
\textbf{Parameters} \\ 
\textit{a} Scalar \\ 
\textbf{Return} \\ 
Reference to the vector\\ 
\end{mdframed}

\begin{mdframed}
Multiply each element of the two vectors
\begin{lstlisting}[language=C++]
Vector& operator*=(const Vector<T,N>& v)
\end{lstlisting}
\textbf{Parameters} \\ 
\textit{v} Vector \\ 
\textbf{Return} \\ 
Reference to the vector\\ 
\textbf{Warning} \\ 
It is not a dot product, use the function dot\\ 
\end{mdframed}

\subsubsection{operator/=}
\begin{mdframed}
Divide each vector element with a scalar
\begin{lstlisting}[language=C++]
Vector& operator/=(T a)
\end{lstlisting}
\textbf{Parameters} \\ 
\textit{a} Scalar \\ 
\textbf{Return} \\ 
Reference to the vector\\ 
\end{mdframed}

\begin{mdframed}
Divide each element of the two vectors
\begin{lstlisting}[language=C++]
Vector& operator/=(const Vector<T,N>& v)
\end{lstlisting}
\textbf{Parameters} \\ 
\textit{v} Vector \\ 
\textbf{Return} \\ 
Reference to the vector\\ 
\end{mdframed}



\subsection{Vector functions}
\subsubsection{operator<<}
\begin{mdframed}
Stream operator, write the vector into a stream
\begin{lstlisting}[language=C++]
template<typename T, uint32_t N> std::ostream& operator<<(std::ostream& stream, const Vector<T,N>& v) 
\end{lstlisting}
\textbf{Template Parameters} \\ 
\textit{T} The type of vector \\ 
\textit{N} The size of the vector \\ 
\textbf{Parameters} \\ 
\textit{stream} The stream \\ 
\textit{v} The vector \\ 
\textbf{Return} \\ 
The stream with the modifications\\ 
\end{mdframed}

\subsubsection{operator+}
\begin{mdframed}
Add a scalar to each element of the vector
\begin{lstlisting}[language=C++]
template<typename T, uint32_t N> Vector<T,N> operator+(Vector<T,N> v1, T a) 
\end{lstlisting}
\textbf{Template Parameters} \\ 
\textit{T} The type of vector \\ 
\textit{N} The size of the vector \\ 
\textbf{Parameters} \\ 
\textit{v1} The vector \\ 
\textit{a} The scalar \\ 
\textbf{Return} \\ 
The new vector\\ 
\end{mdframed}

\begin{mdframed}
Add a scalar to each element of the vector
\begin{lstlisting}[language=C++]
template<typename T, uint32_t N> Vector<T,N> operator+(T a, Vector<T,N> v1) 
\end{lstlisting}
\textbf{Template Parameters} \\ 
\textit{T} The type of vector \\ 
\textit{N} The size of the vector \\ 
\textbf{Parameters} \\ 
\textit{a} The scalar \\ 
\textit{v1} The vector \\ 
\textbf{Return} \\ 
The new vector\\ 
\end{mdframed}

\begin{mdframed}
Addition of two vectors
\begin{lstlisting}[language=C++]
template<typename T, uint32_t N> Vector<T,N> operator+(Vector<T,N> v1, const Vector<T,N>& v2) 
\end{lstlisting}
\textbf{Template Parameters} \\ 
\textit{T} The type of vector \\ 
\textit{N} The size of the vector \\ 
\textbf{Parameters} \\ 
\textit{v1} First vector \\ 
\textit{v2} Second vector \\ 
\textbf{Return} \\ 
The addition of the two vectors\\ 
\end{mdframed}

\subsubsection{operator-}
\begin{mdframed}
Substract a scalar to each element of the vector
\begin{lstlisting}[language=C++]
template<typename T, uint32_t N> Vector<T,N> operator-(Vector<T,N> v1, T a) 
\end{lstlisting}
\textbf{Template Parameters} \\ 
\textit{T} The type of vector \\ 
\textit{N} The size of the vector \\ 
\textbf{Parameters} \\ 
\textit{v1} The vector \\ 
\textit{a} The scalar \\ 
\textbf{Return} \\ 
The new vector\\ 
\end{mdframed}

\begin{mdframed}
Substract element of the vector by a scalar
\begin{lstlisting}[language=C++]
template<typename T, uint32_t N> Vector<T,N> operator-(T a, Vector<T,N> v1) 
\end{lstlisting}
\textbf{Template Parameters} \\ 
\textit{T} The type of vector \\ 
\textit{N} The size of the vector \\ 
\textbf{Parameters} \\ 
\textit{a} The scalar \\ 
\textit{v1} The vector \\ 
\textbf{Return} \\ 
The new vector\\ 
\end{mdframed}

\begin{mdframed}
Substraction of two vectors
\begin{lstlisting}[language=C++]
template<typename T, uint32_t N> Vector<T,N> operator-(Vector<T,N> v1, const Vector<T,N>& v2) 
\end{lstlisting}
\textbf{Template Parameters} \\ 
\textit{T} The type of vector \\ 
\textit{N} The size of the vector \\ 
\textbf{Parameters} \\ 
\textit{v1} First vector \\ 
\textit{v2} Second vector \\ 
\textbf{Return} \\ 
The substraction of the two vectors\\ 
\end{mdframed}

\begin{mdframed}
Each element is multiply by -1
\begin{lstlisting}[language=C++]
template<typename T, uint32_t N> Vector<T,N> operator-(Vector<T,N> v1) 
\end{lstlisting}
\textbf{Template Parameters} \\ 
\textit{T} The type of vector \\ 
\textit{N} The size of the vector \\ 
\textbf{Parameters} \\ 
\textit{v1} The vector \\ 
\textbf{Return} \\ 
Minus the vector (-v)\\ 
\end{mdframed}

\subsubsection{operator*}
\begin{mdframed}
Multiply each component of a vector by a scalar
\begin{lstlisting}[language=C++]
template<typename T, uint32_t N> Vector<T,N> operator*(Vector<T,N> v1, T a) 
\end{lstlisting}
\textbf{Template Parameters} \\ 
\textit{T} The type of vector \\ 
\textit{N} The size of the vector \\ 
\textbf{Parameters} \\ 
\textit{v1} The vector \\ 
\textit{a} The scalar \\ 
\textbf{Return} \\ 
The new vector\\ 
\end{mdframed}

\begin{mdframed}
Multiply each component of a vector by a scalar
\begin{lstlisting}[language=C++]
template<typename T, uint32_t N> Vector<T,N> operator*(T a, Vector<T,N> v1) 
\end{lstlisting}
\textbf{Template Parameters} \\ 
\textit{T} The type of vector \\ 
\textit{N} The size of the vector \\ 
\textbf{Parameters} \\ 
\textit{a} The scalar \\ 
\textit{v1} The vector \\ 
\textbf{Return} \\ 
The new vector\\ 
\end{mdframed}

\begin{mdframed}
Multiply each element of two vectors
\begin{lstlisting}[language=C++]
template<typename T, uint32_t N> Vector<T,N> operator*(Vector<T,N> v1, const Vector<T,N>& v2) 
\end{lstlisting}
\textbf{Template Parameters} \\ 
\textit{T} The type of vector \\ 
\textit{N} The size of the vector \\ 
\textbf{Parameters} \\ 
\textit{v1} First vector \\ 
\textit{v2} Second vector \\ 
\textbf{Return} \\ 
The multiplication result\\ 
\textbf{Warning} \\ 
It is not a dot product, use the function dot\\ 
\end{mdframed}

\subsubsection{operator/}
\begin{mdframed}
Divide each component of a vector by a scalar
\begin{lstlisting}[language=C++]
template<typename T, uint32_t N> Vector<T,N> operator/(Vector<T,N> v1, T a) 
\end{lstlisting}
\textbf{Template Parameters} \\ 
\textit{T} The type of vector \\ 
\textit{N} The size of the vector \\ 
\textbf{Parameters} \\ 
\textit{v1} The vector \\ 
\textit{a} The scalar \\ 
\textbf{Return} \\ 
The new vector\\ 
\end{mdframed}

\begin{mdframed}
Divide each element of two vectors
\begin{lstlisting}[language=C++]
template<typename T, uint32_t N> Vector<T,N> operator/(Vector<T,N> v1, const Vector<T,N>& v2) 
\end{lstlisting}
\textbf{Template Parameters} \\ 
\textit{T} The type of vector \\ 
\textit{N} The size of the vector \\ 
\textbf{Parameters} \\ 
\textit{v1} First vector \\ 
\textit{v2} Second vector \\ 
\textbf{Return} \\ 
The division result\\ 
\end{mdframed}

\subsubsection{sum}
\begin{mdframed}
Sum all the component of a vector
\begin{lstlisting}[language=C++]
template<typename T, uint32_t N> T sum(const Vector<T,N>& v) 
\end{lstlisting}
\textbf{Template Parameters} \\ 
\textit{T} Type of the vector \\ 
\textit{N} Size of the vector \\ 
\textbf{Parameters} \\ 
\textit{v} The vector \\ 
\textbf{Return} \\ 
The sum of the component of the vector\\ 
\end{mdframed}

\subsubsection{merge}
\begin{mdframed}
Merge two vectors into one vector
\begin{lstlisting}[language=C++]
template<typename T, uint32_t N, uint32_t M> Vector<T, N+M> merge(const Vector<T,N>& v1, const Vector<T,M>& v2) 
\end{lstlisting}
\textbf{Template Parameters} \\ 
\textit{T} Type of the vector \\ 
\textit{N} Size of the first vector \\ 
\textit{M} Size of the second vector \\ 
\textbf{Parameters} \\ 
\textit{v1} First vector \\ 
\textit{v2} Second vector \\ 
\textbf{Return} \\ 
Merged vector\\ 
\end{mdframed}

\subsubsection{append}
\begin{mdframed}
Append a component at the back of a vector
\begin{lstlisting}[language=C++]
template<typename T, uint32_t N> Vector<T,N+1> append(const Vector<T,N>& v, T value) 
\end{lstlisting}
\textbf{Template Parameters} \\ 
\textit{T} Type of the vector \\ 
\textit{N} Size of the vector \\ 
\textbf{Parameters} \\ 
\textit{v} The vector \\ 
\textit{value} The new component \\ 
\textbf{Return} \\ 
Vector with the new element\\ 
\end{mdframed}

\subsubsection{insert}
\begin{mdframed}
Insert a component in a vector
\begin{lstlisting}[language=C++]
template<typename T, uint32_t N> Vector<T,N+1> insert(const Vector<T,N>& v, T value, uint32_t index) 
\end{lstlisting}
\textbf{Template Parameters} \\ 
\textit{T} Type of the vector \\ 
\textit{N} Size of the vector \\ 
\textbf{Parameters} \\ 
\textit{v} The vector \\ 
\textit{value} The new component \\ 
\textit{index} The index where the component will be inserted \\ 
\textbf{Return} \\ 
Vector with the new element\\ 
\end{mdframed}

\subsubsection{remove}
\begin{mdframed}
Remove a component from a vector
\begin{lstlisting}[language=C++]
template<typename T, uint32_t N> Vector<T,N-1> remove(const Vector<T,N>&v, uint32_t index) 
\end{lstlisting}
\textbf{Template Parameters} \\ 
\textit{T} Type of the vector \\ 
\textit{N} Size of the vector \\ 
\textbf{Parameters} \\ 
\textit{v} The vector \\ 
\textit{index} The index of the component that will be removed \\ 
\textbf{Return} \\ 
Vector without the component\\ 
\end{mdframed}

\subsubsection{sub\_vector}
\begin{mdframed}
Create a sub vector from a vector
\begin{lstlisting}[language=C++]
template<typename T, uint32_t N, uint32_t I1, uint32_t I2> Vector<T,I2-I1+1> sub_vector(const Vector<T,N>& v) 
\end{lstlisting}
\textbf{Template Parameters} \\ 
\textit{T} Type of the vector \\ 
\textit{N} Size of the vector \\ 
\textit{I1} First element of the sub vector \\ 
\textit{I2} Last element of the sub vector \\ 
\textbf{Parameters} \\ 
\textit{v} The vector \\ 
\textbf{Return} \\ 
The sub vector\\ 
\end{mdframed}

\subsubsection{dot}
\begin{mdframed}
Compute the dot product between two vectors
\begin{lstlisting}[language=C++]
template<typename T, uint32_t N> T dot(Vector<T,N> v1, const Vector<T,N>& v2) 
\end{lstlisting}
\textbf{Template Parameters} \\ 
\textit{T} Type of the vector \\ 
\textit{N} Size of the vector \\ 
\textbf{Parameters} \\ 
\textit{v1} First vector \\ 
\textit{v2} Second vector \\ 
\textbf{Return} \\ 
Result of the dot product\\ 
\end{mdframed}

\subsubsection{cross}
\begin{mdframed}
Compute the cross product between two vectors of dimension 2
\begin{lstlisting}[language=C++]
template<typename T> T cross(const Vector<T,2>& v1, const Vector<T,2>& v2) 
\end{lstlisting}
\textbf{Template Parameters} \\ 
\textit{T} Type of the vector \\ 
\textbf{Parameters} \\ 
\textit{v1} First vector \\ 
\textit{v2} Second vector \\ 
\textbf{Return} \\ 
Result of the cross product (a scalar)\\ 
\end{mdframed}

\begin{mdframed}
Compute the cross product between two vectors of dimension 3
\begin{lstlisting}[language=C++]
template<typename T> Vector<T,3> cross(const Vector<T,3>& v1, const Vector<T,3>& v2) 
\end{lstlisting}
\textbf{Template Parameters} \\ 
\textit{T} Type of the vector \\ 
\textbf{Parameters} \\ 
\textit{v1} First vector \\ 
\textit{v2} Second vector \\ 
\textbf{Return} \\ 
Result of the cross product (a vector 3)\\ 
\end{mdframed}

\subsubsection{angle}
\begin{mdframed}
Compute the angle between two vectors of dimension 2
\begin{lstlisting}[language=C++]
template<typename T> T angle(const Vector<T,2>& v1, const Vector<T,2>& v2) 
\end{lstlisting}
\textbf{Template Parameters} \\ 
\textit{T} Type of the vector \\ 
\textbf{Parameters} \\ 
\textit{v1} First vector \\ 
\textit{v2} Second vector \\ 
\textbf{Return} \\ 
Angle between the vectors\\ 
\end{mdframed}

\begin{mdframed}
Compute the angle between two vectors of dimension 3
\begin{lstlisting}[language=C++]
template<typename T> T angle(const Vector<T,3>& v1, const Vector<T,3>& v2) 
\end{lstlisting}
\textbf{Template Parameters} \\ 
\textit{T} Type of the vector \\ 
\textbf{Parameters} \\ 
\textit{v1} First vector \\ 
\textit{v2} Second vector \\ 
\textbf{Return} \\ 
Angle between the vectors\\ 
\end{mdframed}

\subsubsection{polar}
\begin{mdframed}
Compute the polar representation of a vector 2
\begin{lstlisting}[language=C++]
template<typename T> void polar(const Vector<T,2>& v, T& radius, T& theta) 
\end{lstlisting}
\textbf{Template Parameters} \\ 
\textit{T} Type of the vector \\ 
\textbf{Parameters} \\ 
\textit{v} The vector \\ 
\textit{radius} A reference to the magnitude \\ 
\textit{theta} A reference to the argument (angle) of the vector \\ 
\end{mdframed}

\subsubsection{cylindrical}
\begin{mdframed}
Compute the cylindrical representation of a vector 3
\begin{lstlisting}[language=C++]
template<typename T> void cylindrical(const Vector<T,3>& v, T& radius, T& theta, T& z) 
\end{lstlisting}
\textbf{Template Parameters} \\ 
\textit{T} Type of the vector \\ 
\textbf{Parameters} \\ 
\textit{v} The vector \\ 
\textit{radius} A reference to the radius of the cylinder \\ 
\textit{theta} A reference to the angle \\ 
\textit{z} A reference to the z coordinate \\ 
\end{mdframed}

\subsubsection{spherical}
\begin{mdframed}
Compute the spherical representation of a vector 3
\begin{lstlisting}[language=C++]
template<typename T> void spherical(const Vector<T,3>& v, T& radius, T& theta, T& phi) 
\end{lstlisting}
\textbf{Template Parameters} \\ 
\textit{T} Type of the vector \\ 
\textbf{Parameters} \\ 
\textit{v} The vector \\ 
\textit{radius} A reference to the radius of the sphere \\ 
\textit{theta} A reference to the theta angle \\ 
\textit{phi} A reference to the phi angle \\ 
\end{mdframed}

\subsubsection{distance}
\begin{mdframed}
Compute the distance between two vectors
\begin{lstlisting}[language=C++]
template<typename T, uint32_t N> T distance(const Vector<T,N>& v1, const Vector<T,N>& v2) 
\end{lstlisting}
\textbf{Template Parameters} \\ 
\textit{T} Type of vector \\ 
\textit{N} Size of vectors \\ 
\textbf{Parameters} \\ 
\textit{v1} First vector \\ 
\textit{v2} Second vector \\ 
\textbf{Return} \\ 
The distance between the vectors\\ 
\end{mdframed}

\subsubsection{reflect}
\begin{mdframed}
Compute the reflected vector
\begin{lstlisting}[language=C++]
template<typename T, uint32_t N> Vector<T,N> reflect(const Vector<T,N>& incident, Vector<T,N> surface_normal) 
\end{lstlisting}
\textbf{Template Parameters} \\ 
\textit{T} Type of vector \\ 
\textit{N} Site of vector \\ 
\textbf{Parameters} \\ 
\textit{incident} The incident ray \\ 
\textit{surface\_normal} The surface normal \\ 
\textbf{Return} \\ 
The refleted vector\\ 
\end{mdframed}

\subsubsection{refract}
\begin{mdframed}
Compute the refracted vector
\begin{lstlisting}[language=C++]
template<typename T, uint32_t N> Vector<T,N> refract(const Vector<T,N>& incident, Vector<T,N> surface_normal, T n1, T n2) 
\end{lstlisting}
\textbf{Template Parameters} \\ 
\textit{T} Type of vector \\ 
\textit{N} Size of vector \\ 
\textbf{Parameters} \\ 
\textit{incident} The incident ray \\ 
\textit{surface\_normal} The surface normal \\ 
\textit{n1} The refraction index of the medium of the incident ray \\ 
\textit{n2} The refraction index of the medium of the surface \\ 
\textbf{Return} \\ 
The refracted vector\\ 
\end{mdframed}

\subsubsection{face\_same\_direction}
\begin{mdframed}
Get if both vector are facing in the same direction (dot(v1, v2) >? 0)
\begin{lstlisting}[language=C++]
template<typename T, uint32_t N> bool face_same_direction(const Vector<T,N>& v1, const Vector<T,N>& v2) 
\end{lstlisting}
\textbf{Template Parameters} \\ 
\textit{T} Type of vector \\ 
\textit{N} Size of vector \\ 
\textbf{Parameters} \\ 
\textit{v1} First vector \\ 
\textit{v2} Second vector \\ 
\textbf{Return} \\ 
True if dot(v1, v2) > 0, false otherwise\\ 
\end{mdframed}



\subsection{Matrix  class}
Mathematical matrix
\begin{lstlisting}[language=C++]
template<typename T, uint32_t N, uint32_t M> class Matrix 
\end{lstlisting}
\textbf{Template Parameters} \\ 
\textit{T} Type of matrix \\ 
\textit{N} Row \\ 
\textit{M} Column \\ 

\subsubsection{Matrix}
\begin{mdframed}
Default constructor, fill the matri with value 0
\begin{lstlisting}[language=C++]
Matrix()
\end{lstlisting}
\end{mdframed}

\begin{mdframed}
Constructor, fill the matrix with a value
\begin{lstlisting}[language=C++]
Matrix(T fill_val)
\end{lstlisting}
\textbf{Parameters} \\ 
\textit{fill\_val} Value that fill the matrix \\ 
\end{mdframed}

\begin{mdframed}
Constructor, copy the data
\begin{lstlisting}[language=C++]
Matrix(T* start)
\end{lstlisting}
\textbf{Parameters} \\ 
\textit{start} First element to copy \\ 
\end{mdframed}

\begin{mdframed}
Constructor, copy an initializer list of vector
\begin{lstlisting}[language=C++]
Matrix(const std::initializer_list<Vector<T,N>>& init_list)
\end{lstlisting}
\textbf{Parameters} \\ 
\textit{l} Initializer list of vector \\ 
\end{mdframed}

\begin{mdframed}
Constructor, copy a initializer list
\begin{lstlisting}[language=C++]
Matrix(const std::initializer_list<T>& init_list)
\end{lstlisting}
\textbf{Parameters} \\ 
\textit{l} Initializer list \\ 
\end{mdframed}

\begin{mdframed}
Constructor, copy an array
\begin{lstlisting}[language=C++]
Matrix(const std::array<T, N*M>& init_array)
\end{lstlisting}
\textbf{Parameters} \\ 
\textit{l} Initial array \\ 
\end{mdframed}

\begin{mdframed}
Constructor, copy an array of vectors
\begin{lstlisting}[language=C++]
Matrix(const std::array<Vector<T, N>, M>& init_array)
\end{lstlisting}
\textbf{Parameters} \\ 
\textit{l} Array of vectors \\ 
\end{mdframed}

\subsubsection{get\_type\_object}
\begin{mdframed}
Get the type of the object
\begin{lstlisting}[language=C++]
virtual type_mat get_type_object() const override 
\end{lstlisting}
\textbf{Return} \\ 
The type of the object\\ 
\end{mdframed}

\subsubsection{size}
\begin{mdframed}
Get the size of the matrix
\begin{lstlisting}[language=C++]
void size(uint32_t& n, uint32_t& m) const 
\end{lstlisting}
\textbf{Parameters} \\ 
\textit{n} Reference to the number of row \\ 
\textit{m} Reference to the number of column \\ 
\end{mdframed}

\subsubsection{copy}
\begin{mdframed}
Copy data into the matrix
\begin{lstlisting}[language=C++]
void copy(T* start)
\end{lstlisting}
\textbf{Parameters} \\ 
\textit{start} A pointer to the first element that will be copy \\ 
\end{mdframed}

\subsubsection{row}
\begin{mdframed}
Get the number of row
\begin{lstlisting}[language=C++]
uint32_t row() const 
\end{lstlisting}
\textbf{Return} \\ 
The number of row\\ 
\end{mdframed}

\subsubsection{column}
\begin{mdframed}
Get the number of column
\begin{lstlisting}[language=C++]
uint32_t column() const 
\end{lstlisting}
\textbf{Return} \\ 
The number of column\\ 
\end{mdframed}

\subsubsection{begin}
\begin{mdframed}
Get the begin iterator
\begin{lstlisting}[language=C++]
iterator begin() 
\end{lstlisting}
\textbf{Return} \\ 
The begin iterator\\ 
\end{mdframed}

\begin{mdframed}
Get the constant begin iterator
\begin{lstlisting}[language=C++]
const_iterator begin() const 
\end{lstlisting}
\textbf{Return} \\ 
The constant begin iterator\\ 
\end{mdframed}

\subsubsection{end}
\begin{mdframed}
Get the end iterator
\begin{lstlisting}[language=C++]
iterator end() 
\end{lstlisting}
\textbf{Return} \\ 
The end iterator\\ 
\end{mdframed}

\begin{mdframed}
Get the constant end iterator
\begin{lstlisting}[language=C++]
const_iterator end() const 
\end{lstlisting}
\textbf{Return} \\ 
The constant end iterator\\ 
\end{mdframed}

\subsubsection{operator()}
\begin{mdframed}
Accessor to the element (i,j)
\begin{lstlisting}[language=C++]
const T& operator()(uint32_t i, uint32_t j) const 
\end{lstlisting}
\textbf{Parameters} \\ 
\textit{i} Row \\ 
\textit{j} Column \\ 
\textbf{Return} \\ 
A constant reference to the element (i,j) \\ 
\end{mdframed}

\begin{mdframed}
Accessor to the element (i,j)
\begin{lstlisting}[language=C++]
T& operator()(uint32_t i, uint32_t j) 
\end{lstlisting}
\textbf{Parameters} \\ 
\textit{i} Row \\ 
\textit{j} Column \\ 
\textbf{Return} \\ 
A reference to the element (i,j) \\ 
\end{mdframed}

\subsubsection{operator[]}
\begin{mdframed}
Accessor to the i th column vector
\begin{lstlisting}[language=C++]
const Vector<T,N>& operator[](uint32_t i) const 
\end{lstlisting}
\textbf{Parameters} \\ 
\textit{i} Column \\ 
\textbf{Return} \\ 
A constant reference to the column vector i\\ 
\end{mdframed}

\begin{mdframed}
Accessor to the i th column vector
\begin{lstlisting}[language=C++]
Vector<T,N>& operator[](uint32_t i) 
\end{lstlisting}
\textbf{Parameters} \\ 
\textit{i} Column \\ 
\textbf{Return} \\ 
A reference to the column vector i\\ 
\end{mdframed}

\subsubsection{operator+=}
\begin{mdframed}
Sum each component with a scalar
\begin{lstlisting}[language=C++]
Matrix& operator+=(T a)
\end{lstlisting}
\textbf{Parameters} \\ 
\textit{a} Scalar \\ 
\textbf{Return} \\ 
Reference to the matrix\\ 
\end{mdframed}

\begin{mdframed}
Add two matrices
\begin{lstlisting}[language=C++]
Matrix& operator+=(const Matrix<T,N,M>& v)
\end{lstlisting}
\textbf{Parameters} \\ 
\textit{v} Matrix \\ 
\textbf{Return} \\ 
Reference to the matrix\\ 
\end{mdframed}

\subsubsection{operator-=}
\begin{mdframed}
Substract each component with a scalar
\begin{lstlisting}[language=C++]
Matrix& operator-=(T a)
\end{lstlisting}
\textbf{Parameters} \\ 
\textit{a} Scalar \\ 
\textbf{Return} \\ 
Reference to the matrix\\ 
\end{mdframed}

\begin{mdframed}
Substract two matrices
\begin{lstlisting}[language=C++]
Matrix& operator-=(const Matrix<T,N,M>& v)
\end{lstlisting}
\textbf{Parameters} \\ 
\textit{v} Matrix \\ 
\textbf{Return} \\ 
Reference to the matrix\\ 
\end{mdframed}

\subsubsection{operator*=}
\begin{mdframed}
Multiply each component with a scalar
\begin{lstlisting}[language=C++]
Matrix& operator*=(T a)
\end{lstlisting}
\textbf{Parameters} \\ 
\textit{a} Scalar \\ 
\textbf{Return} \\ 
Reference to the matrix\\ 
\end{mdframed}

\begin{mdframed}
Multiply each element of the two matrices
\begin{lstlisting}[language=C++]
Matrix& operator*=(const Matrix<T,N,M>& v)
\end{lstlisting}
\textbf{Parameters} \\ 
\textit{v} Matrix \\ 
\textbf{Return} \\ 
Reference to the matrix\\ 
\textbf{Warning} \\ 
It is not a matrix product, use the function dot\\ 
\end{mdframed}

\subsubsection{operator/=}
\begin{mdframed}
Divide each component with a scalar
\begin{lstlisting}[language=C++]
Matrix& operator/=(T a)
\end{lstlisting}
\textbf{Parameters} \\ 
\textit{a} Scalar \\ 
\textbf{Return} \\ 
Reference to the matrix\\ 
\end{mdframed}

\begin{mdframed}
Divide each element of the two matrices
\begin{lstlisting}[language=C++]
Matrix& operator/=(const Matrix<T,N,M>& v)
\end{lstlisting}
\textbf{Parameters} \\ 
\textit{v} Matrix \\ 
\textbf{Return} \\ 
Reference to the matrix\\ 
\end{mdframed}



\subsection{Matrix functions}
\subsubsection{operator<<}
\begin{mdframed}
Stream operator, write the matrix into a stream
\begin{lstlisting}[language=C++]
template<typename T, uint32_t N, uint32_t M> std::ostream& operator<<(std::ostream& stream, const Matrix<T,N,M>& m) 
\end{lstlisting}
\textbf{Template Parameters} \\ 
\textit{T} The type of matrix \\ 
\textit{N} Row \\ 
\textit{M} Column \\ 
\textbf{Parameters} \\ 
\textit{stream} The stream \\ 
\textit{m} The matrix \\ 
\textbf{Return} \\ 
The stream with the modifications\\ 
\end{mdframed}

\subsubsection{operator+}
\begin{mdframed}
Add a scalar to each element of the matrix
\begin{lstlisting}[language=C++]
template<typename T, uint32_t N, uint32_t M> Matrix<T,N,M> operator+(Matrix<T,N,M> m1, T a) 
\end{lstlisting}
\textbf{Template Parameters} \\ 
\textit{T} The type of matrix \\ 
\textit{N} Row \\ 
\textit{M} Column \\ 
\textbf{Parameters} \\ 
\textit{m1} The matrix \\ 
\textit{a} The scalar \\ 
\textbf{Return} \\ 
The new matrix\\ 
\end{mdframed}

\begin{mdframed}
Add a scalar to each element of the matrix
\begin{lstlisting}[language=C++]
template<typename T, uint32_t N, uint32_t M> Matrix<T,N,M> operator+(T a, Matrix<T,N,M> m1) 
\end{lstlisting}
\textbf{Template Parameters} \\ 
\textit{T} The type of matrix \\ 
\textit{N} Row \\ 
\textit{M} Column \\ 
\textbf{Parameters} \\ 
\textit{a} The scalar \\ 
\textit{m1} The matrix \\ 
\textbf{Return} \\ 
The new matrix\\ 
\end{mdframed}

\begin{mdframed}
Addition of two matrices
\begin{lstlisting}[language=C++]
template<typename T, uint32_t N, uint32_t M> Matrix<T,N,M> operator+(Matrix<T,N,M> m1, const Matrix<T,N,M>& m2) 
\end{lstlisting}
\textbf{Template Parameters} \\ 
\textit{T} The type of matrix \\ 
\textit{N} Row \\ 
\textit{M} Column \\ 
\textbf{Parameters} \\ 
\textit{m1} The first matrix \\ 
\textit{m2} The second matrix \\ 
\textbf{Return} \\ 
The addition of the matrices\\ 
\end{mdframed}

\subsubsection{operator-}
\begin{mdframed}
Substract a scalar to each element of the matrix
\begin{lstlisting}[language=C++]
template<typename T, uint32_t N, uint32_t M> Matrix<T,N,M> operator-(Matrix<T,N,M> m1, T a) 
\end{lstlisting}
\textbf{Template Parameters} \\ 
\textit{T} The type of matrix \\ 
\textit{N} Row \\ 
\textit{M} Column \\ 
\textbf{Parameters} \\ 
\textit{m1} The matrix \\ 
\textit{a} The scalar \\ 
\textbf{Return} \\ 
The new matrix\\ 
\end{mdframed}

\begin{mdframed}
Substract element of the matrix by a scalar
\begin{lstlisting}[language=C++]
template<typename T, uint32_t N, uint32_t M> Matrix<T,N,M> operator-(T a, Matrix<T,N,M> m1) 
\end{lstlisting}
\textbf{Template Parameters} \\ 
\textit{T} The type of matrix \\ 
\textit{N} Row \\ 
\textit{M} Column \\ 
\textbf{Parameters} \\ 
\textit{a} The scalar \\ 
\textit{m1} The matrix \\ 
\textbf{Return} \\ 
The new matrix\\ 
\end{mdframed}

\begin{mdframed}
Substraction of two matrices
\begin{lstlisting}[language=C++]
template<typename T, uint32_t N, uint32_t M> Matrix<T,N,M> operator-(Matrix<T,N,M> m1, const Matrix<T,N,M>& m2) 
\end{lstlisting}
\textbf{Template Parameters} \\ 
\textit{T} The type of matrix \\ 
\textit{N} Row \\ 
\textit{M} Column \\ 
\textbf{Parameters} \\ 
\textit{m1} The first matrix \\ 
\textit{m2} The second matrix \\ 
\textbf{Return} \\ 
The substraction of the matrices\\ 
\end{mdframed}

\subsubsection{operator*}
\begin{mdframed}
Multiply each component of a matrix by a scalar
\begin{lstlisting}[language=C++]
template<typename T, uint32_t N, uint32_t M> Matrix<T,N,M> operator*(Matrix<T,N,M> m1, T a) 
\end{lstlisting}
\textbf{Template Parameters} \\ 
\textit{T} The type of matrix \\ 
\textit{N} Row \\ 
\textit{M} Column \\ 
\textbf{Parameters} \\ 
\textit{m1} The matrix \\ 
\textit{a} The scalar \\ 
\textbf{Return} \\ 
The new matrix\\ 
\end{mdframed}

\begin{mdframed}
Multiply each component of a matrix by a scalar
\begin{lstlisting}[language=C++]
template<typename T, uint32_t N, uint32_t M> Matrix<T,N,M> operator*(T a, Matrix<T,N,M> m1) 
\end{lstlisting}
\textbf{Template Parameters} \\ 
\textit{T} The type of matrix \\ 
\textit{N} Row \\ 
\textit{M} Column \\ 
\textbf{Parameters} \\ 
\textit{a} The scalar \\ 
\textit{m1} The matrix \\ 
\textbf{Return} \\ 
The new matrix\\ 
\end{mdframed}

\begin{mdframed}
Multiplication of each elements of two matrices
\begin{lstlisting}[language=C++]
template<typename T, uint32_t N, uint32_t M> Matrix<T,N,M> operator*(Matrix<T,N,M> m1, const Matrix<T,N,M>& m2) 
\end{lstlisting}
\textbf{Template Parameters} \\ 
\textit{T} The type of matrix \\ 
\textit{N} Row \\ 
\textit{M} Column \\ 
\textbf{Parameters} \\ 
\textit{m1} The first matrix \\ 
\textit{m2} The second matrix \\ 
\textbf{Return} \\ 
The multiplication of each element of the matrices\\ 
\textbf{Warning} \\ 
This is not the matrix product. Use dot function for the matrix product\\ 
\end{mdframed}

\subsubsection{operator/}
\begin{mdframed}
Divide each component of a matrix by a scalar
\begin{lstlisting}[language=C++]
template<typename T, uint32_t N, uint32_t M> Matrix<T,N,M> operator/(Matrix<T,N,M> m1, T a) 
\end{lstlisting}
\textbf{Template Parameters} \\ 
\textit{T} The type of matrix \\ 
\textit{N} Row \\ 
\textit{M} Column \\ 
\textbf{Parameters} \\ 
\textit{m1} The matrix \\ 
\textit{a} The scalar \\ 
\textbf{Return} \\ 
The new matrix\\ 
\end{mdframed}

\begin{mdframed}
Division of each elements of two matrices
\begin{lstlisting}[language=C++]
template<typename T, uint32_t N, uint32_t M> Matrix<T,N,M> operator/(Matrix<T,N,M> m1, const Matrix<T,N,M>& m2) 
\end{lstlisting}
\textbf{Template Parameters} \\ 
\textit{T} The type of matrix \\ 
\textit{N} Row \\ 
\textit{M} Column \\ 
\textbf{Parameters} \\ 
\textit{m1} The first matrix \\ 
\textit{m2} The second matrix \\ 
\textbf{Return} \\ 
The division of each element of the matrices\\ 
\textbf{Warning} \\ 
This is not the matrix product. Use dot function for the matrix product\\ 
\end{mdframed}

\subsubsection{min}
\begin{mdframed}
Get the minimum value inside the matrix
\begin{lstlisting}[language=C++]
template<typename T, uint32_t N, uint32_t M> T min(const Matrix<T,N,M>& m) 
\end{lstlisting}
\textbf{Template Parameters} \\ 
\textit{T} The type of matrix \\ 
\textit{N} Row \\ 
\textit{M} Column \\ 
\textbf{Parameters} \\ 
\textit{m} The matrix \\ 
\textbf{Return} \\ 
The minimum component in the matrix\\ 
\end{mdframed}

\begin{mdframed}
Compare each component with a value and return a matrix containing the minimum between component and value
\begin{lstlisting}[language=C++]
template<typename T, uint32_t N, uint32_t M> Matrix<T,N,M> min(Matrix<T,N,M> m, T a) 
\end{lstlisting}
\textbf{Template Parameters} \\ 
\textit{T} The type of matrix \\ 
\textit{N} Row \\ 
\textit{M} Column \\ 
\textbf{Parameters} \\ 
\textit{m} The matrix \\ 
\textit{a} The value \\ 
\textbf{Return} \\ 
A matrix containing the minimum between the component and the value\\ 
\end{mdframed}

\begin{mdframed}
Compare each component and return a matrix containing the minimum value in each component
\begin{lstlisting}[language=C++]
template<typename T, uint32_t N, uint32_t M> Matrix<T,N,M> min(Matrix<T,N,M> m1, const Matrix<T,N,M>& m2) 
\end{lstlisting}
\textbf{Template Parameters} \\ 
\textit{T} The type of matrix \\ 
\textit{N} Row \\ 
\textit{M} Column \\ 
\textbf{Parameters} \\ 
\textit{m1} The first matrix \\ 
\textit{m2} The second matrix \\ 
\textbf{Return} \\ 
A matrix containing the minimum between the component of each matrix\\ 
\end{mdframed}

\subsubsection{max}
\begin{mdframed}
Get the maximum value inside the matrix
\begin{lstlisting}[language=C++]
template<typename T, uint32_t N, uint32_t M> T max(const Matrix<T,N,M>& m) 
\end{lstlisting}
\textbf{Template Parameters} \\ 
\textit{T} The type of matrix \\ 
\textit{N} Row \\ 
\textit{M} Column \\ 
\textbf{Parameters} \\ 
\textit{m} The matrix \\ 
\textbf{Return} \\ 
The maximum component in the matrix\\ 
\end{mdframed}

\begin{mdframed}
Compare each component with a value and return a matrix containing the maximum between component and value
\begin{lstlisting}[language=C++]
template<typename T, uint32_t N, uint32_t M> Matrix<T,N,M> max(Matrix<T,N,M> m, T a) 
\end{lstlisting}
\textbf{Template Parameters} \\ 
\textit{T} The type of matrix \\ 
\textit{N} Row \\ 
\textit{M} Column \\ 
\textbf{Parameters} \\ 
\textit{m} The matrix \\ 
\textit{a} The value \\ 
\textbf{Return} \\ 
A matrix containing the maximum between the component and the value\\ 
\end{mdframed}

\begin{mdframed}
Compare each component and return a matrix containing the maximum value in each component
\begin{lstlisting}[language=C++]
template<typename T, uint32_t N, uint32_t M> Matrix<T,N,M> max(Matrix<T,N,M> m1, const Matrix<T,N,M>& m2) 
\end{lstlisting}
\textbf{Template Parameters} \\ 
\textit{T} The type of matrix \\ 
\textit{N} Row \\ 
\textit{M} Column \\ 
\textbf{Parameters} \\ 
\textit{m1} The first matrix \\ 
\textit{m2} The second matrix \\ 
\textbf{Return} \\ 
A matrix containing the maximum between the component of each matrix\\ 
\end{mdframed}

\subsubsection{abs}
\begin{mdframed}
Apply absolute value to each component of a matrix
\begin{lstlisting}[language=C++]
template<typename T, uint32_t N, uint32_t M> Matrix<T,N,M> abs(Matrix<T,N,M> m) 
\end{lstlisting}
\textbf{Template Parameters} \\ 
\textit{T} The type of matrix \\ 
\textit{N} Row \\ 
\textit{M} Column \\ 
\textbf{Parameters} \\ 
\textit{m} The matrix \\ 
\textbf{Return} \\ 
A matrix with absolute value apply to each component\\ 
\end{mdframed}

\subsubsection{sign}
\begin{mdframed}
Apply sign function to each component of a matrix
\begin{lstlisting}[language=C++]
template<typename T, uint32_t N, uint32_t M> Matrix<T,N,M> sign(Matrix<T,N,M> m) 
\end{lstlisting}
\textbf{Template Parameters} \\ 
\textit{T} The type of matrix \\ 
\textit{N} Row \\ 
\textit{M} Column \\ 
\textbf{Parameters} \\ 
\textit{m} The matrix \\ 
\textbf{Return} \\ 
A matrix with sign function apply to each component\\ 
\end{mdframed}

\subsubsection{to\_row\_matrix}
\begin{mdframed}
Transform a vector to a row matrix
\begin{lstlisting}[language=C++]
template<typename T, uint32_t N> Matrix<T,1,N> to_row_matrix(const Vector<T,N>& v) 
\end{lstlisting}
\textbf{Template Parameters} \\ 
\textit{T} Type \\ 
\textit{N} Size \\ 
\textbf{Parameters} \\ 
\textit{v} The vector \\ 
\textbf{Return} \\ 
The row matrix\\ 
\end{mdframed}

\subsubsection{to\_column\_matrix}
\begin{mdframed}
Transform a vector to a column matrix
\begin{lstlisting}[language=C++]
template<typename T, uint32_t N> Matrix<T,N,1> to_column_matrix(const Vector<T,N>& v) 
\end{lstlisting}
\textbf{Template Parameters} \\ 
\textit{T} Type \\ 
\textit{N} Size \\ 
\textbf{Parameters} \\ 
\textit{v} The vector \\ 
\textbf{Return} \\ 
The column matrix\\ 
\end{mdframed}

\subsubsection{to\_vector}
\begin{mdframed}
Transform a column matrix to a vector
\begin{lstlisting}[language=C++]
template<typename T, uint32_t N> Vector<T,N> to_vector(const Matrix<T,N,1>& m) 
\end{lstlisting}
\textbf{Template Parameters} \\ 
\textit{T} Type \\ 
\textit{N} Size \\ 
\textbf{Parameters} \\ 
\textit{m} The column matrix \\ 
\textbf{Return} \\ 
The vector\\ 
\end{mdframed}

\begin{mdframed}
Transform a row matrix to a vector
\begin{lstlisting}[language=C++]
template<typename T, uint32_t N> Vector<T,N> to_vector(const Matrix<T,1,N>& m) 
\end{lstlisting}
\textbf{Template Parameters} \\ 
\textit{T} Type \\ 
\textit{N} Size \\ 
\textbf{Parameters} \\ 
\textit{m} The row matrix \\ 
\textbf{Return} \\ 
The vector\\ 
\end{mdframed}

\subsubsection{dot}
\begin{mdframed}
Matrix product
\begin{lstlisting}[language=C++]
template<typename T, uint32_t N, uint32_t M, uint32_t P> Matrix<T,N,P> dot(const Matrix<T,N,M>& m1, const Matrix<T,M,P>& m2) 
\end{lstlisting}
\textbf{Template Parameters} \\ 
\textit{T} Type \\ 
\textit{N} Row of the first matrix \\ 
\textit{M} Column of the first matrix and row of the second matrix \\ 
\textit{P} Column of the second matrix \\ 
\textbf{Parameters} \\ 
\textit{m1} The first matrix \\ 
\textit{m2} The second matrix \\ 
\textbf{Return} \\ 
The result matrix\\ 
\end{mdframed}

\begin{mdframed}
Matrix - vector product
\begin{lstlisting}[language=C++]
template<typename T, uint32_t N, uint32_t M> BaseVector<T,N> dot(const Matrix<T,N,M>& m, const BaseVector<T,M>& v) 
\end{lstlisting}
\textbf{Template Parameters} \\ 
\textit{T} Type \\ 
\textit{N} Row of the matrix \\ 
\textit{M} Column of the matrix and size of the vector \\ 
\textbf{Parameters} \\ 
\textit{m} The matrix \\ 
\textit{v} The vector \\ 
\textbf{Return} \\ 
The result vector\\ 
\end{mdframed}

\begin{mdframed}
Vector - matrix product
\begin{lstlisting}[language=C++]
template<typename T, uint32_t N, uint32_t M> BaseVector<T,M> dot(const BaseVector<T,N>& v, const Matrix<T,N,M>& m) 
\end{lstlisting}
\textbf{Template Parameters} \\ 
\textit{T} Type \\ 
\textit{N} Row of the matrix and size of the vector \\ 
\textit{M} Column of the matrix \\ 
\textbf{Parameters} \\ 
\textit{v} The vector \\ 
\textit{m} The matrix \\ 
\textbf{Return} \\ 
The result vector\\ 
\end{mdframed}

\subsubsection{identity}
\begin{mdframed}
Get the identity matrix
\begin{lstlisting}[language=C++]
template<typename T, uint32_t N> Matrix<T,N,N> identity() 
\end{lstlisting}
\textbf{Template Parameters} \\ 
\textit{T} Type \\ 
\textit{N} Row and column of the matrix \\ 
\textbf{Return} \\ 
The identity matrix\\ 
\end{mdframed}

\subsubsection{transpose}
\begin{mdframed}
Get the transpose of a matrix
\begin{lstlisting}[language=C++]
template<typename T, uint32_t N, uint32_t M> Matrix<T,M,N> transpose(const Matrix<T,N,M>& m) 
\end{lstlisting}
\textbf{Template Parameters} \\ 
\textit{T} Type \\ 
\textit{N} Row of the matrix \\ 
\textit{M} Column of the matrix \\ 
\textbf{Parameters} \\ 
\textit{m} The matrix \\ 
\textbf{Return} \\ 
The transpose matrix\\ 
\end{mdframed}

\subsubsection{self\_transpose}
\begin{mdframed}
Transpose the matrix itself
\begin{lstlisting}[language=C++]
template<typename T, uint32_t N> void self_transpose(Matrix<T,N,N>& m) 
\end{lstlisting}
\textbf{Template Parameters} \\ 
\textit{T} Type \\ 
\textit{N} Row and column of the matrix \\ 
\textbf{Parameters} \\ 
\textit{m} The matrix \\ 
\end{mdframed}

\subsubsection{direct\_sum}
\begin{mdframed}
Direct sum of two matrices
\begin{lstlisting}[language=C++]
template<typename T, uint32_t N, uint32_t M, uint32_t P, uint32_t Q> Matrix<T,N+P,M+Q> direct_sum(const Matrix<T,N,M>& m1, const Matrix<T,P,Q>& m2) 
\end{lstlisting}
\textbf{Template Parameters} \\ 
\textit{T} Type \\ 
\textit{N} Row of the first matrix \\ 
\textit{M} Column of the first matrix \\ 
\textit{P} Row of the second matrix \\ 
\textit{Q} Column of the second matrix \\ 
\textbf{Parameters} \\ 
\textit{m1} The first matrix \\ 
\textit{m2} The second matrix \\ 
\textbf{Return} \\ 
The direct sum of the matrices\\ 
\end{mdframed}

\subsubsection{kronecker\_product}
\begin{mdframed}
Kronecker product of two matrices
\begin{lstlisting}[language=C++]
template<typename T, uint32_t N, uint32_t M, uint32_t P, uint32_t Q> Matrix<T,N*P,M*Q> kronecker_product(const Matrix<T,N,M>& m1, const Matrix<T,P,Q>& m2) 
\end{lstlisting}
\textbf{Template Parameters} \\ 
\textit{T} Type \\ 
\textit{N} Row of the first matrix \\ 
\textit{M} Column of the first matrix \\ 
\textit{P} Row of the second matrix \\ 
\textit{Q} Column of the second matrix \\ 
\textbf{Parameters} \\ 
\textit{m1} The first matrix \\ 
\textit{m2} The second matrix \\ 
\textbf{Return} \\ 
The kronecker product of the matrices\\ 
\end{mdframed}

\subsubsection{determinant}
\begin{mdframed}
Compute the determinant of a matrix
\begin{lstlisting}[language=C++]
template<typename T, uint32_t N> T determinant(Matrix<T,N,N> m) 
\end{lstlisting}
\textbf{Template Parameters} \\ 
\textit{T} Type \\ 
\textit{N} Row and column of the first matrix \\ 
\textbf{Parameters} \\ 
\textit{m} The matrix \\ 
\textbf{Return} \\ 
The determinant of the matrix\\ 
\end{mdframed}

\subsubsection{inverse}
\begin{mdframed}
Compute the inverse of a matrix
\begin{lstlisting}[language=C++]
template<typename T, uint32_t N> Matrix<T,N,N> inverse(Matrix<T,N,N> m) 
\end{lstlisting}
\textbf{Template Parameters} \\ 
\textit{T} Type \\ 
\textit{N} Row and column of the first matrix \\ 
\textbf{Parameters} \\ 
\textit{m} The matrix \\ 
\textbf{Return} \\ 
The inverse of the matrix\\ 
\textbf{Warning} \\ 
Does not check if the determinant is non null\\ 
\end{mdframed}

\subsubsection{expand}
\begin{mdframed}
Expand a matrix (from [N x N] to [N+1 x N+1])
\begin{lstlisting}[language=C++]
template<typename T, uint32_t N> Matrix<T,N+1,N+1> expand(const Matrix<T,N,N>& m) 
\end{lstlisting}
\textbf{Template Parameters} \\ 
\textit{T} Type \\ 
\textit{N} Row and column of the first matrix \\ 
\textbf{Parameters} \\ 
\textit{m} The matrix \\ 
\textbf{Return} \\ 
The expand matrix\\ 
\end{mdframed}

\subsubsection{exp}
\begin{mdframed}
Compute the exponential of a matrix
\begin{lstlisting}[language=C++]
template<typename T, uint32_t N> Matrix<T,N,N> exp(const Matrix<T,N,N>& m, T tolerence = 1e-3, uint32_t max_iteration = 30) 
\end{lstlisting}
\textbf{Template Parameters} \\ 
\textit{T} Type of matrix \\ 
\textit{N} Row and column \\ 
\textbf{Parameters} \\ 
\textit{m} The matrix \\ 
\textit{tolerence} Tolerence of the computation \\ 
\textit{max\_iteration} Maximum number of iterations \\ 
\textbf{Return} \\ 
The exponential of a matrix\\ 
\textbf{Warning} \\ 
Tolerence must be smaller than 1, max\_iteration must be smaller than 33\\ 
\end{mdframed}



\subsection{Complex  class}
Complex number
\begin{lstlisting}[language=C++]
template<typename T> class Complex 
\end{lstlisting}
\textbf{Template Parameters} \\ 
\textit{T} Type of complex \\ 

\subsubsection{Complex}
\begin{mdframed}
Default constructor, fill the vector with value 0
\begin{lstlisting}[language=C++]
Complex()
\end{lstlisting}
\end{mdframed}

\begin{mdframed}
Constructor with only a real part
\begin{lstlisting}[language=C++]
Complex(T re)
\end{lstlisting}
\textbf{Parameters} \\ 
\textit{re} Real part \\ 
\end{mdframed}

\begin{mdframed}
Constructor with a real and an imaginary part
\begin{lstlisting}[language=C++]
Complex(T re, T im)
\end{lstlisting}
\textbf{Parameters} \\ 
\textit{re} Real part \\ 
\textit{im} Imaginary part \\ 
\end{mdframed}

\begin{mdframed}
Constructor, copy an array
\begin{lstlisting}[language=C++]
Complex(const std::array<T,2>& init_array)
\end{lstlisting}
\textbf{Parameters} \\ 
\textit{init\_array} Initial array \\ 
\end{mdframed}

\begin{mdframed}
Constructor, copy an initializer list
\begin{lstlisting}[language=C++]
Complex(const std::initializer_list<T>& init_list)
\end{lstlisting}
\textbf{Parameters} \\ 
\textit{init\_list} Initializer list \\ 
\end{mdframed}

\subsubsection{get\_type\_object}
\begin{mdframed}
Get the type of the object
\begin{lstlisting}[language=C++]
virtual type_mat get_type_object() const override 
\end{lstlisting}
\textbf{Return} \\ 
The type of the object\\ 
\end{mdframed}

\subsubsection{real}
\begin{mdframed}
Get the real part
\begin{lstlisting}[language=C++]
T& real()
\end{lstlisting}
\textbf{Return} \\ 
A reference to the real part\\ 
\end{mdframed}

\begin{mdframed}
Get the real part
\begin{lstlisting}[language=C++]
const T& real() const
\end{lstlisting}
\textbf{Return} \\ 
A constant reference to the real part\\ 
\end{mdframed}

\subsubsection{imag}
\begin{mdframed}
Get the imaginary part
\begin{lstlisting}[language=C++]
T& imag()
\end{lstlisting}
\textbf{Return} \\ 
A reference to the imaginary part\\ 
\end{mdframed}

\begin{mdframed}
Get the imaginary part
\begin{lstlisting}[language=C++]
const T& imag() const
\end{lstlisting}
\textbf{Return} \\ 
A constant reference to the imaginary part\\ 
\end{mdframed}

\subsubsection{modulus2}
\begin{mdframed}
Compute the square modulus
\begin{lstlisting}[language=C++]
T modulus2() const
\end{lstlisting}
\textbf{Return} \\ 
The square modulus\\ 
\end{mdframed}

\subsubsection{modulus}
\begin{mdframed}
Compute the modulus
\begin{lstlisting}[language=C++]
T modulus() const
\end{lstlisting}
\textbf{Return} \\ 
The modulus\\ 
\end{mdframed}

\subsubsection{argument}
\begin{mdframed}
Compute the argument
\begin{lstlisting}[language=C++]
T argument() const
\end{lstlisting}
\textbf{Return} \\ 
The argument\\ 
\end{mdframed}

\subsubsection{conjugate}
\begin{mdframed}
Conjugate the complex
\begin{lstlisting}[language=C++]
void conjugate()
\end{lstlisting}
\end{mdframed}

\subsubsection{normalize}
\begin{mdframed}
Normalize the complex
\begin{lstlisting}[language=C++]
void normalize()
\end{lstlisting}
\end{mdframed}

\subsubsection{set\_polar}
\begin{mdframed}
Set the component from a polar form
\begin{lstlisting}[language=C++]
void set_polar(T mod, T argu)
\end{lstlisting}
\textbf{Parameters} \\ 
\textit{mod} The modulus of the complex \\ 
\textit{argu} The argument of the complex \\ 
\end{mdframed}

\subsubsection{get\_polar}
\begin{mdframed}
Get the component from a polar form
\begin{lstlisting}[language=C++]
void get_polar(T& magn, T& argu)
\end{lstlisting}
\textbf{Parameters} \\ 
\textit{mod} A reference to the modulus of the complex \\ 
\textit{argu} A reference to the argument of the complex \\ 
\end{mdframed}

\subsubsection{operator BaseVector<T,2>\&}
\begin{mdframed}
Cast to a base vector
\begin{lstlisting}[language=C++]
operator BaseVector<T,2>&()
\end{lstlisting}
\end{mdframed}

\subsubsection{operator+=}
\begin{mdframed}
Add a real number
\begin{lstlisting}[language=C++]
Complex& operator+=(T a)
\end{lstlisting}
\textbf{Parameters} \\ 
\textit{a} The real number \\ 
\textbf{Return} \\ 
Reference to the complex\\ 
\end{mdframed}

\begin{mdframed}
Add two complex
\begin{lstlisting}[language=C++]
Complex& operator+=(const Complex<T>& c)
\end{lstlisting}
\textbf{Parameters} \\ 
\textit{c} Complex \\ 
\textbf{Return} \\ 
Reference to the complex\\ 
\end{mdframed}

\subsubsection{operator-=}
\begin{mdframed}
Substract a real number
\begin{lstlisting}[language=C++]
Complex& operator-=(T a)
\end{lstlisting}
\textbf{Parameters} \\ 
\textit{a} The real number  \\ 
\textbf{Return} \\ 
Reference to the complex\\ 
\end{mdframed}

\begin{mdframed}
Substract two complex
\begin{lstlisting}[language=C++]
Complex& operator-=(const Complex<T>& c)
\end{lstlisting}
\textbf{Parameters} \\ 
\textit{c} Complex \\ 
\textbf{Return} \\ 
Reference to the complex\\ 
\end{mdframed}

\subsubsection{operator*=}
\begin{mdframed}
Multiply by a real number
\begin{lstlisting}[language=C++]
Complex& operator*=(T a)
\end{lstlisting}
\textbf{Parameters} \\ 
\textit{a} The real number  \\ 
\textbf{Return} \\ 
Reference to the complex\\ 
\end{mdframed}

\begin{mdframed}
Multiply two complex
\begin{lstlisting}[language=C++]
Complex& operator*=(const Complex<T>& c)
\end{lstlisting}
\textbf{Parameters} \\ 
\textit{c} Complex \\ 
\textbf{Return} \\ 
Reference to the complex\\ 
\end{mdframed}

\subsubsection{operator/=}
\begin{mdframed}
Divide by a real number
\begin{lstlisting}[language=C++]
Complex& operator/=(T a)
\end{lstlisting}
\textbf{Parameters} \\ 
\textit{a} The real number  \\ 
\textbf{Return} \\ 
Reference to the complex\\ 
\end{mdframed}

\begin{mdframed}
Divide two complex
\begin{lstlisting}[language=C++]
Complex& operator/=(const Complex<T>& c)
\end{lstlisting}
\textbf{Parameters} \\ 
\textit{c} Complex \\ 
\textbf{Return} \\ 
Reference to the complex\\ 
\end{mdframed}



\subsection{Complex functions}
\subsubsection{operator<<}
\begin{mdframed}
Stream operator, write the complex into a stream
\begin{lstlisting}[language=C++]
template<typename T> std::ostream& operator<<(std::ostream& stream, const Complex<T>& c) 
\end{lstlisting}
\textbf{Template Parameters} \\ 
\textit{T} The type of complex \\ 
\textbf{Parameters} \\ 
\textit{stream} The stream \\ 
\textit{c} The complex \\ 
\textbf{Return} \\ 
The stream with the modifications\\ 
\end{mdframed}

\subsubsection{operator+}
\begin{mdframed}
Add a real number to a complex
\begin{lstlisting}[language=C++]
template<typename T> Complex<T> operator+(Complex<T> c, T a) 
\end{lstlisting}
\textbf{Template Parameters} \\ 
\textit{T} The type of complex \\ 
\textbf{Parameters} \\ 
\textit{c} The complex \\ 
\textit{a} The real number \\ 
\textbf{Return} \\ 
The new complex\\ 
\end{mdframed}

\begin{mdframed}
Add a real number to a complex
\begin{lstlisting}[language=C++]
template<typename T> Complex<T> operator+(T a, Complex<T> c) 
\end{lstlisting}
\textbf{Template Parameters} \\ 
\textit{T} The type of complex \\ 
\textbf{Parameters} \\ 
\textit{a} The real number \\ 
\textit{c} The complex \\ 
\textbf{Return} \\ 
The new complex\\ 
\end{mdframed}

\begin{mdframed}
Addition of two complex
\begin{lstlisting}[language=C++]
template<typename T> Complex<T> operator+(Complex<T> c1, const Complex<T>& c2) 
\end{lstlisting}
\textbf{Template Parameters} \\ 
\textit{T} The type of complex \\ 
\textbf{Parameters} \\ 
\textit{c1} First complex \\ 
\textit{c2} Second complex \\ 
\textbf{Return} \\ 
The addition of the two complex\\ 
\end{mdframed}

\subsubsection{operator-}
\begin{mdframed}
Substract a real number to a complex
\begin{lstlisting}[language=C++]
template<typename T> Complex<T> operator-(Complex<T> c, T a) 
\end{lstlisting}
\textbf{Template Parameters} \\ 
\textit{T} The type of complex \\ 
\textbf{Parameters} \\ 
\textit{c} The complex \\ 
\textit{a} The real number \\ 
\textbf{Return} \\ 
The new complex\\ 
\end{mdframed}

\begin{mdframed}
Substract a complex to a real number
\begin{lstlisting}[language=C++]
template<typename T> Complex<T> operator-(T a, Complex<T> c) 
\end{lstlisting}
\textbf{Template Parameters} \\ 
\textit{T} The type of complex \\ 
\textbf{Parameters} \\ 
\textit{a} The real number \\ 
\textit{c} The complex \\ 
\textbf{Return} \\ 
The new complex\\ 
\end{mdframed}

\begin{mdframed}
Substraction of two complex
\begin{lstlisting}[language=C++]
template<typename T> Complex<T> operator-(Complex<T> c1, const Complex<T>& c2) 
\end{lstlisting}
\textbf{Template Parameters} \\ 
\textit{T} The type of complex \\ 
\textbf{Parameters} \\ 
\textit{c1} First complex \\ 
\textit{c2} Second complex \\ 
\textbf{Return} \\ 
The substraction of the two complex\\ 
\end{mdframed}

\begin{mdframed}
Multiply the complex by -1
\begin{lstlisting}[language=C++]
template<typename T> Complex<T> operator-(const Complex<T>& c) 
\end{lstlisting}
\textbf{Template Parameters} \\ 
\textit{T} The type of complex \\ 
\textbf{Parameters} \\ 
\textit{c} The complex \\ 
\textbf{Return} \\ 
Minus the complex (-z)\\ 
\end{mdframed}

\subsubsection{operator*}
\begin{mdframed}
Multiply a complex by a real number
\begin{lstlisting}[language=C++]
template<typename T> Complex<T> operator*(Complex<T> c, T a) 
\end{lstlisting}
\textbf{Template Parameters} \\ 
\textit{T} The type of complex \\ 
\textbf{Parameters} \\ 
\textit{c} The complex \\ 
\textit{a} The real number \\ 
\textbf{Return} \\ 
The new complex\\ 
\end{mdframed}

\begin{mdframed}
Multiply a complex by a real number
\begin{lstlisting}[language=C++]
template<typename T> Complex<T> operator*(T a, Complex<T> c) 
\end{lstlisting}
\textbf{Template Parameters} \\ 
\textit{T} The type of complex \\ 
\textbf{Parameters} \\ 
\textit{a} The real number \\ 
\textit{c} The complex \\ 
\textbf{Return} \\ 
The new complex\\ 
\end{mdframed}

\begin{mdframed}
Multiplication of two complex
\begin{lstlisting}[language=C++]
template<typename T> Complex<T> operator*(Complex<T> c1, const Complex<T>& c2) 
\end{lstlisting}
\textbf{Template Parameters} \\ 
\textit{T} The type of complex \\ 
\textbf{Parameters} \\ 
\textit{c1} First complex \\ 
\textit{c2} Second complex \\ 
\textbf{Return} \\ 
The multiplication of the two complex\\ 
\end{mdframed}

\subsubsection{operator/}
\begin{mdframed}
Divide a complex by a real number
\begin{lstlisting}[language=C++]
template<typename T> Complex<T> operator/(Complex<T> c, T a) 
\end{lstlisting}
\textbf{Template Parameters} \\ 
\textit{T} The type of complex \\ 
\textbf{Parameters} \\ 
\textit{c} The complex \\ 
\textit{a} The real number \\ 
\textbf{Return} \\ 
The newcomplex\\ 
\end{mdframed}

\begin{mdframed}
Divide a real number by a complex
\begin{lstlisting}[language=C++]
template<typename T> Complex<T> operator/(T a, const Complex<T>& c) 
\end{lstlisting}
\textbf{Template Parameters} \\ 
\textit{T} The type of complex \\ 
\textbf{Parameters} \\ 
\textit{a} The real number \\ 
\textit{c} The complex \\ 
\textbf{Return} \\ 
The new complex\\ 
\end{mdframed}

\begin{mdframed}
Division of two complex
\begin{lstlisting}[language=C++]
template<typename T> Complex<T> operator/(Complex<T> c1, const Complex<T>& c2) 
\end{lstlisting}
\textbf{Template Parameters} \\ 
\textit{T} The type of complex \\ 
\textbf{Parameters} \\ 
\textit{c1} First complex \\ 
\textit{c2} Second complex \\ 
\textbf{Return} \\ 
The division of the two complex\\ 
\end{mdframed}

\subsubsection{complex\_to\_rotation\_matrix}
\begin{mdframed}
Compute the rotation matrix from a complex
\begin{lstlisting}[language=C++]
template<typename T> Matrix<T,2,2> complex_to_rotation_matrix(const Complex<T>& c) 
\end{lstlisting}
\textbf{Template Parameters} \\ 
\textit{T} The type of complex \\ 
\textbf{Parameters} \\ 
\textit{c} The complex \\ 
\textbf{Return} \\ 
The rotation matrix\\ 
\end{mdframed}

\subsubsection{angle\_to\_complex}
\begin{mdframed}
Compute the complex associate to a rotation of an angle
\begin{lstlisting}[language=C++]
template<typename T> Complex<T> angle_to_complex(T angle) 
\end{lstlisting}
\textbf{Template Parameters} \\ 
\textit{T} The type of complex \\ 
\textbf{Parameters} \\ 
\textit{angle} The angle of rotation \\ 
\textbf{Return} \\ 
The complex\\ 
\end{mdframed}

\subsubsection{complex\_rotate\_vec}
\begin{mdframed}
Rotate a vector using a complex as rotation
\begin{lstlisting}[language=C++]
template<typename T> Vector<T,2> complex_rotate_vec(const Complex<T>& c, Vector<T,2> v) 
\end{lstlisting}
\textbf{Template Parameters} \\ 
\textit{T} Type of complex \\ 
\textbf{Parameters} \\ 
\textit{c} The type complex \\ 
\textit{v} The vector \\ 
\textbf{Return} \\ 
The rotated vector\\ 
\end{mdframed}

\subsubsection{conjugate}
\begin{mdframed}
Compute the conjugate of a complex
\begin{lstlisting}[language=C++]
template<typename T> Complex<T> conjugate(Complex<T> c) 
\end{lstlisting}
\textbf{Template Parameters} \\ 
\textit{T} The type of complex \\ 
\textbf{Parameters} \\ 
\textit{c} The complex \\ 
\textbf{Return} \\ 
The conjugate of the complex\\ 
\end{mdframed}

\begin{mdframed}
Compute the conjugate of a complex matrix
\begin{lstlisting}[language=C++]
template<typename T, uint32_t N, uint32_t M> Matrix<Complex<T>, N, M> conjugate(Matrix<Complex<T>, N, M> m) 
\end{lstlisting}
\textbf{Template Parameters} \\ 
\textit{T} The type of complex \\ 
\textit{N} Row of the matrix \\ 
\textit{M} Column of the matrix \\ 
\textbf{Parameters} \\ 
\textit{m} The complex matrix \\ 
\textbf{Return} \\ 
The conjugate of the matrix\\ 
\end{mdframed}

\subsubsection{self\_conjugate}
\begin{mdframed}
Conjugate a complex matrix
\begin{lstlisting}[language=C++]
template<typename T, uint32_t N, uint32_t M> void self_conjugate(Matrix<Complex<T>, N, M>& m) 
\end{lstlisting}
\textbf{Template Parameters} \\ 
\textit{T} Type of complex \\ 
\textit{N} Row of the matrix \\ 
\textit{M} Column of the matrix \\ 
\textbf{Parameters} \\ 
\textit{m} The complex matrix \\ 
\end{mdframed}

\subsubsection{conjugate\_transpose}
\begin{mdframed}
Compute the conjugate transpose of a complex matrix
\begin{lstlisting}[language=C++]
template<typename T, uint32_t N, uint32_t M> Matrix<Complex<T>, M, N> conjugate_transpose(const Matrix<Complex<T>, N, M>& m) 
\end{lstlisting}
\textbf{Template Parameters} \\ 
\textit{T} The type of complex \\ 
\textit{N} Row of the matrix \\ 
\textit{M} Column of the matrix \\ 
\textbf{Parameters} \\ 
\textit{m} The complex matrix \\ 
\textbf{Return} \\ 
The conjugate transpose of the complex matrix\\ 
\end{mdframed}

\subsubsection{self\_conjugate\_transpose}
\begin{mdframed}
Conjugate transpose a complex matrix
\begin{lstlisting}[language=C++]
template<typename T, uint32_t N> void self_conjugate_transpose(Matrix<Complex<T>, N, N>& m) 
\end{lstlisting}
\textbf{Template Parameters} \\ 
\textit{T} The type of complex \\ 
\textit{N} The row and column of the matrix \\ 
\textbf{Parameters} \\ 
\textit{m} The complex matrix \\ 
\end{mdframed}



\subsection{Quaternion  class}
Quaternion
\begin{lstlisting}[language=C++]
template<typename T> class Quaternion 
\end{lstlisting}
\textbf{Template Parameters} \\ 
\textit{T} Type of quaternion \\ 

\subsubsection{Quaternion}
\begin{mdframed}
Default constructor, quaternion is zero
\begin{lstlisting}[language=C++]
Quaternion()
\end{lstlisting}
\end{mdframed}

\begin{mdframed}
Constructor, quaternion is equal to a real number
\begin{lstlisting}[language=C++]
Quaternion(T real)
\end{lstlisting}
\textbf{Parameters} \\ 
\textit{real} The real number \\ 
\end{mdframed}

\begin{mdframed}
Constructor, use a real part (scalar) and an imaginary part (vector 3)
\begin{lstlisting}[language=C++]
Quaternion(T real, const Vector<T,3>& imag)
\end{lstlisting}
\textbf{Parameters} \\ 
\textit{real} Real part (scalar) \\ 
\textit{imag} Imaginary part (vector 3) \\ 
\end{mdframed}

\begin{mdframed}
Constructor, use a real part (scalar) and an imaginary part (array 3)
\begin{lstlisting}[language=C++]
Quaternion(T real, const std::array<T,3>& imag)
\end{lstlisting}
\textbf{Parameters} \\ 
\textit{real} Real part (scalar) \\ 
\textit{imag} Imaginary part (array 3) \\ 
\end{mdframed}

\begin{mdframed}
Constructor, fill each component real, i, j, k
\begin{lstlisting}[language=C++]
Quaternion(T real, T i, T j, T k)
\end{lstlisting}
\textbf{Parameters} \\ 
\textit{real} Real component \\ 
\textit{i} i component \\ 
\textit{j} j component \\ 
\textit{k} k component \\ 
\end{mdframed}

\begin{mdframed}
Constructor, use an initial array 4
\begin{lstlisting}[language=C++]
Quaternion(const std::array<T,4>& init_array)
\end{lstlisting}
\textbf{Parameters} \\ 
\textit{init\_array} Initial array \\ 
\end{mdframed}

\begin{mdframed}
Constructor, use an initializer list
\begin{lstlisting}[language=C++]
Quaternion(const std::initializer_list<T>& init_list)
\end{lstlisting}
\textbf{Parameters} \\ 
\textit{init\_list} Initializer list \\ 
\end{mdframed}

\subsubsection{get\_type\_object}
\begin{mdframed}
Get the type of the object
\begin{lstlisting}[language=C++]
virtual type_mat get_type_object() const override 
\end{lstlisting}
\textbf{Return} \\ 
The type of the object\\ 
\end{mdframed}

\subsubsection{r}
\begin{mdframed}
Get a reference to the real part
\begin{lstlisting}[language=C++]
T& r()
\end{lstlisting}
\textbf{Return} \\ 
A reference to the real part\\ 
\end{mdframed}

\begin{mdframed}
Get a constant reference to the real part
\begin{lstlisting}[language=C++]
const T& r() const
\end{lstlisting}
\textbf{Return} \\ 
A constant reference to the real part\\ 
\end{mdframed}

\subsubsection{i}
\begin{mdframed}
Get a reference to the i part
\begin{lstlisting}[language=C++]
T& i()
\end{lstlisting}
\textbf{Return} \\ 
A reference to the i part\\ 
\end{mdframed}

\begin{mdframed}
Get a constant reference to the i part
\begin{lstlisting}[language=C++]
const T& i() const
\end{lstlisting}
\textbf{Return} \\ 
A constant reference to the i part\\ 
\end{mdframed}

\subsubsection{j}
\begin{mdframed}
Get a reference to the j part
\begin{lstlisting}[language=C++]
T& j()
\end{lstlisting}
\textbf{Return} \\ 
A reference to the j part\\ 
\end{mdframed}

\begin{mdframed}
Get a constant reference to the j part
\begin{lstlisting}[language=C++]
const T& j() const
\end{lstlisting}
\textbf{Return} \\ 
A constant reference to the j part\\ 
\end{mdframed}

\subsubsection{k}
\begin{mdframed}
Get a reference to the k part
\begin{lstlisting}[language=C++]
T& k()
\end{lstlisting}
\textbf{Return} \\ 
A reference to the k part\\ 
\end{mdframed}

\begin{mdframed}
Get a constant reference to the k part
\begin{lstlisting}[language=C++]
const T& k() const
\end{lstlisting}
\textbf{Return} \\ 
A constant reference to the k part\\ 
\end{mdframed}

\subsubsection{norm2}
\begin{mdframed}
Compute the square of the norm
\begin{lstlisting}[language=C++]
T norm2() const
\end{lstlisting}
\textbf{Return} \\ 
The square of the norm\\ 
\end{mdframed}

\subsubsection{norm}
\begin{mdframed}
Compute the norm
\begin{lstlisting}[language=C++]
T norm() const
\end{lstlisting}
\textbf{Return} \\ 
The norm\\ 
\end{mdframed}

\subsubsection{normalize}
\begin{mdframed}
Normalize the quaternion
\begin{lstlisting}[language=C++]
void normalize()
\end{lstlisting}
\end{mdframed}

\subsubsection{conjugate}
\begin{mdframed}
Conjugate the quaternion
\begin{lstlisting}[language=C++]
void conjugate()
\end{lstlisting}
\end{mdframed}

\subsubsection{real}
\begin{mdframed}
Get a copy of the real component
\begin{lstlisting}[language=C++]
T real() const
\end{lstlisting}
\textbf{Return} \\ 
A copy of the real component\\ 
\end{mdframed}

\subsubsection{imag}
\begin{mdframed}
Get a copy of the imaginary components
\begin{lstlisting}[language=C++]
Vector<T,3> imag() const
\end{lstlisting}
\textbf{Return} \\ 
A vector 3 of the imaginary components\\ 
\end{mdframed}

\subsubsection{operator BaseVector<T,4>\&}
\begin{mdframed}
Cast to a base vector
\begin{lstlisting}[language=C++]
operator BaseVector<T,4>&()
\end{lstlisting}
\end{mdframed}

\subsubsection{operator+=}
\begin{mdframed}
Add a real number
\begin{lstlisting}[language=C++]
Quaternion& operator+=(T a)
\end{lstlisting}
\textbf{Parameters} \\ 
\textit{a} The real number \\ 
\textbf{Return} \\ 
Reference to quaternion\\ 
\end{mdframed}

\begin{mdframed}
Add a quaternion
\begin{lstlisting}[language=C++]
Quaternion& operator+=(const Quaternion<T>& q)
\end{lstlisting}
\textbf{Parameters} \\ 
\textit{q} The quaternion \\ 
\textbf{Return} \\ 
Reference to quaternion\\ 
\end{mdframed}

\subsubsection{operator-=}
\begin{mdframed}
Substract a real number
\begin{lstlisting}[language=C++]
Quaternion& operator-=(T a)
\end{lstlisting}
\textbf{Parameters} \\ 
\textit{a} The real number \\ 
\textbf{Return} \\ 
Reference to quaternion\\ 
\end{mdframed}

\begin{mdframed}
Substract a quaternion
\begin{lstlisting}[language=C++]
Quaternion& operator-=(const Quaternion<T>& q)
\end{lstlisting}
\textbf{Parameters} \\ 
\textit{q} The quaternion \\ 
\textbf{Return} \\ 
Reference to quaternion\\ 
\end{mdframed}

\subsubsection{operator*=}
\begin{mdframed}
Multiply the quaternion by a real number
\begin{lstlisting}[language=C++]
Quaternion& operator*=(T a)
\end{lstlisting}
\textbf{Parameters} \\ 
\textit{a} The real number \\ 
\textbf{Return} \\ 
Reference to quaternion\\ 
\end{mdframed}

\begin{mdframed}
Multiply a quaternion
\begin{lstlisting}[language=C++]
Quaternion& operator*=(const Quaternion<T>& q)
\end{lstlisting}
\textbf{Parameters} \\ 
\textit{q} The quaternion \\ 
\textbf{Return} \\ 
Reference to quaternion\\ 
\end{mdframed}

\subsubsection{operator/=}
\begin{mdframed}
Divide the quaternion by a real number
\begin{lstlisting}[language=C++]
Quaternion& operator/=(T a)
\end{lstlisting}
\textbf{Parameters} \\ 
\textit{a} The real number \\ 
\textbf{Return} \\ 
Reference to quaternion\\ 
\end{mdframed}



\subsection{Quaternion functions}
\subsubsection{operator<<}
\begin{mdframed}
Stream operator, write the quaternion into a stream
\begin{lstlisting}[language=C++]
template<typename T> std::ostream& operator<<(std::ostream& stream, const Quaternion<T>& q) 
\end{lstlisting}
\textbf{Template Parameters} \\ 
\textit{T} Type of quaternion \\ 
\textbf{Parameters} \\ 
\textit{stream} The stream \\ 
\textit{q} The quaternion \\ 
\textbf{Return} \\ 
The stream with the modifications\\ 
\end{mdframed}

\subsubsection{operator+}
\begin{mdframed}
Add a real to a quaternion
\begin{lstlisting}[language=C++]
template<typename T> Quaternion<T> operator+(Quaternion<T> q, T a) 
\end{lstlisting}
\textbf{Template Parameters} \\ 
\textit{T} Type of quaternion \\ 
\textbf{Parameters} \\ 
\textit{q} The quaternion \\ 
\textit{a} The real \\ 
\textbf{Return} \\ 
The new quaternion\\ 
\end{mdframed}

\begin{mdframed}
Add a real to a quaternion
\begin{lstlisting}[language=C++]
template<typename T> Quaternion<T> operator+(T a, Quaternion<T> q) 
\end{lstlisting}
\textbf{Template Parameters} \\ 
\textit{T} Type of quaternion \\ 
\textbf{Parameters} \\ 
\textit{a} The real \\ 
\textit{q} The quaternion \\ 
\textbf{Return} \\ 
The new quaternion\\ 
\end{mdframed}

\begin{mdframed}
Addition of two quaternions
\begin{lstlisting}[language=C++]
template<typename T> Quaternion<T> operator+(Quaternion<T> q1, const Quaternion<T>& q2) 
\end{lstlisting}
\textbf{Template Parameters} \\ 
\textit{T} Type of quaternion \\ 
\textbf{Parameters} \\ 
\textit{q1} First quaternion \\ 
\textit{q2} Second quaternion \\ 
\textbf{Return} \\ 
The addition of the quaternion\\ 
\end{mdframed}

\subsubsection{operator-}
\begin{mdframed}
Substract a real to a quaternion
\begin{lstlisting}[language=C++]
template<typename T> Quaternion<T> operator-(Quaternion<T> q, T a) 
\end{lstlisting}
\textbf{Template Parameters} \\ 
\textit{T} Type of quaternion \\ 
\textbf{Parameters} \\ 
\textit{q} The quaternion \\ 
\textit{a} The real \\ 
\textbf{Return} \\ 
The new quaternion\\ 
\end{mdframed}

\begin{mdframed}
Substract a quaternion to a real
\begin{lstlisting}[language=C++]
template<typename T> Quaternion<T> operator-(T a, Quaternion<T> q) 
\end{lstlisting}
\textbf{Template Parameters} \\ 
\textit{T} Type of quaternion \\ 
\textbf{Parameters} \\ 
\textit{a} The real \\ 
\textit{q} The quaternion \\ 
\textbf{Return} \\ 
The new quaternion\\ 
\end{mdframed}

\begin{mdframed}
Substraction of two quaternions
\begin{lstlisting}[language=C++]
template<typename T> Quaternion<T> operator-(Quaternion<T> q1, const Quaternion<T>& q2) 
\end{lstlisting}
\textbf{Template Parameters} \\ 
\textit{T} Type of quaternion \\ 
\textbf{Parameters} \\ 
\textit{q1} First quaternion \\ 
\textit{q2} Second quaternion \\ 
\textbf{Return} \\ 
The substraction of the quaternion\\ 
\end{mdframed}

\begin{mdframed}
Multiply the quaternion by -1
\begin{lstlisting}[language=C++]
template<typename T> Quaternion<T> operator-(const Quaternion<T>& q) 
\end{lstlisting}
\textbf{Template Parameters} \\ 
\textit{T} Type of quaternion \\ 
\textbf{Parameters} \\ 
\textit{q} The quaternion \\ 
\textbf{Return} \\ 
Minus the quaternion (-q)\\ 
\end{mdframed}

\subsubsection{operator*}
\begin{mdframed}
Multiply a quaternion by a real
\begin{lstlisting}[language=C++]
template<typename T> Quaternion<T> operator*(Quaternion<T> q, T a) 
\end{lstlisting}
\textbf{Template Parameters} \\ 
\textit{T} Type of quaternion \\ 
\textbf{Parameters} \\ 
\textit{q} The quaternion \\ 
\textit{a} The real \\ 
\textbf{Return} \\ 
The new quaternion\\ 
\end{mdframed}

\begin{mdframed}
Multiply a quaternion by a real
\begin{lstlisting}[language=C++]
template<typename T> Quaternion<T> operator*(T a, Quaternion<T> q) 
\end{lstlisting}
\textbf{Template Parameters} \\ 
\textit{T} Type of quaternion \\ 
\textbf{Parameters} \\ 
\textit{a} The real \\ 
\textit{q} The quaternion \\ 
\textbf{Return} \\ 
The new quaternion\\ 
\end{mdframed}

\begin{mdframed}
Multiplication of two quaternions
\begin{lstlisting}[language=C++]
template<typename T> Quaternion<T> operator*(Quaternion<T> q1, const Quaternion<T>& q2) 
\end{lstlisting}
\textbf{Template Parameters} \\ 
\textit{T} Type of quaternion \\ 
\textbf{Parameters} \\ 
\textit{q1} First quaternion \\ 
\textit{q2} Second quaternion \\ 
\textbf{Return} \\ 
The multiplication of the quaternion\\ 
\end{mdframed}

\subsubsection{operator/}
\begin{mdframed}
Divide a quaternion by a real
\begin{lstlisting}[language=C++]
template<typename T> Quaternion<T> operator/(Quaternion<T> q, T a) 
\end{lstlisting}
\textbf{Template Parameters} \\ 
\textit{T} Type of quaternion \\ 
\textbf{Parameters} \\ 
\textit{q} The quaternion \\ 
\textit{a} The real \\ 
\textbf{Return} \\ 
The new quaternion\\ 
\end{mdframed}

\subsubsection{quat\_to\_rotation3}
\begin{mdframed}
From a quaternion to a rotation matrix (3x3)
\begin{lstlisting}[language=C++]
template<typename T> Matrix<T,3,3> quat_to_rotation3(const Quaternion<T>& q) 
\end{lstlisting}
\textbf{Template Parameters} \\ 
\textit{T} Type of quaternion \\ 
\textbf{Parameters} \\ 
\textit{q} The quaternion \\ 
\textbf{Return} \\ 
The rotation matrix (3x3)\\ 
\end{mdframed}

\subsubsection{quat\_to\_rotation4}
\begin{mdframed}
From a quaternion to a rotation matrix (4x4)
\begin{lstlisting}[language=C++]
template<typename T> Matrix<T,4,4> quat_to_rotation4(const Quaternion<T>& q) 
\end{lstlisting}
\textbf{Template Parameters} \\ 
\textit{T} Type of quaternion \\ 
\textbf{Parameters} \\ 
\textit{q} The quaternion \\ 
\textbf{Return} \\ 
The rotation matrix (4x4) (last line and column fill with 0 except the diagonal term which is 1)\\ 
\end{mdframed}

\subsubsection{quat\_to\_vec\_of\_rotation}
\begin{mdframed}
Get the vector of the rotation
\begin{lstlisting}[language=C++]
template<typename T> Vector<T,3> quat_to_vec_of_rotation(const Quaternion<T>& q) 
\end{lstlisting}
\textbf{Template Parameters} \\ 
\textit{T} Type of quaternion \\ 
\textbf{Parameters} \\ 
\textit{q} The quaternion \\ 
\textbf{Return} \\ 
The vector of rotation\\ 
\end{mdframed}

\subsubsection{quat\_to\_angle\_of\_rotation}
\begin{mdframed}
Get the angle of rotation
\begin{lstlisting}[language=C++]
template<typename T> T quat_to_angle_of_rotation(const Quaternion<T>& q) 
\end{lstlisting}
\textbf{Template Parameters} \\ 
\textit{T} Type of quaternion \\ 
\textbf{Parameters} \\ 
\textit{q} The quaternion \\ 
\textbf{Return} \\ 
The angle of rotation\\ 
\end{mdframed}

\subsubsection{quat\_from\_angle\_vec\_of\_rotation}
\begin{mdframed}
Compute a quaternion from a vector of rotation and  an angle
\begin{lstlisting}[language=C++]
template<typename T> Quaternion<T> quat_from_angle_vec_of_rotation(T angle, Vector<T,3> vec) 
\end{lstlisting}
\textbf{Template Parameters} \\ 
\textit{T} Type of quaternion \\ 
\textbf{Parameters} \\ 
\textit{angle} The angle \\ 
\textit{vec} The vector of rotation \\ 
\textbf{Return} \\ 
The quaternion that represent the rotation\\ 
\end{mdframed}

\subsubsection{euler\_angle\_to\_quat}
\begin{mdframed}
Compute the quaternion from the Euler angles
\begin{lstlisting}[language=C++]
template<typename T> Quaternion<T> euler_angle_to_quat(T roll, T pitch, T yaw) 
\end{lstlisting}
\textbf{Template Parameters} \\ 
\textit{T} Type of quaternion \\ 
\textbf{Parameters} \\ 
\textit{roll} Roll angle (around x) \\ 
\textit{pitch} Pitch angle (around y) \\ 
\textit{yaw} Yaw angle (around z) \\ 
\textbf{Return} \\ 
The quaternion that represent the rotation\\ 
\end{mdframed}

\subsubsection{quat\_to\_euler\_angle}
\begin{mdframed}
Get the Euler angles from a quaternion
\begin{lstlisting}[language=C++]
template<typename T> void quat_to_euler_angle(const Quaternion<T>& q, T& roll, T& pitch, T& yaw) 
\end{lstlisting}
\textbf{Template Parameters} \\ 
\textit{T} Type of quaternion \\ 
\textbf{Parameters} \\ 
\textit{q} The quaternion \\ 
\textit{roll} Reference to the roll angle (around x) \\ 
\textit{pitch} Reference to the pitch angle (around y) \\ 
\textit{yaw} Reference to the yaw angle (around z) \\ 
\end{mdframed}

\subsubsection{quat\_rotate\_vec}
\begin{mdframed}
Rotate a vector using a quaternion
\begin{lstlisting}[language=C++]
template<typename T> Vector<T,3> quat_rotate_vec(const Quaternion<T>& q, const Vector<T,3>& v) 
\end{lstlisting}
\textbf{Template Parameters} \\ 
\textit{T} Type of quaternion and vector \\ 
\textbf{Parameters} \\ 
\textit{q} The quaternion \\ 
\textit{v} The vector \\ 
\textbf{Return} \\ 
The rotated vector\\ 
\end{mdframed}

\subsubsection{conjugate}
\begin{mdframed}
Conjugate the quaternion
\begin{lstlisting}[language=C++]
template<typename T> Quaternion<T> conjugate(Quaternion<T> q) 
\end{lstlisting}
\textbf{Template Parameters} \\ 
\textit{T} Type of quaternion \\ 
\textbf{Parameters} \\ 
\textit{q} The quaternion \\ 
\textbf{Return} \\ 
The conjugated quaternion\\ 
\end{mdframed}

\subsubsection{inverse}
\begin{mdframed}
Inverse the quaternion
\begin{lstlisting}[language=C++]
template<typename T> Quaternion<T> inverse(const Quaternion<T>& q) 
\end{lstlisting}
\textbf{Template Parameters} \\ 
\textit{T} Type of quaternion \\ 
\textbf{Parameters} \\ 
\textit{q} The quaternion \\ 
\textbf{Return} \\ 
The inversed quaternion\\ 
\end{mdframed}







\end{document}
